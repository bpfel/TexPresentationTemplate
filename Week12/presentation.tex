%===========================================================================
%generation of assignment and solution in one .tex file. make sure that write18 is activated(--enable-write18 for example as you call pdflatex.exe)
%\gdef\conditionmacro{2}
\ifx\conditionmacro\undefined 
\immediate\write18{%
	pdfLaTeX --jobname="presentation_handout"
	\gdef\string\conditionmacro{1}\string\input\space\jobname
}%
\immediate\write18{%
	pdfLaTeX --jobname="presentation_with_notes"
	\gdef\string\conditionmacro{3}\string\input\space\jobname
}%
\gdef\conditionmacro{2}
%\expandafter\stop
\fi
%---------------------------------------------------------------------------

\ifnum\conditionmacro=1 \documentclass[handout,usenames,dvipsnames]{beamer}\fi
\ifnum\conditionmacro=2 \documentclass[usenames,dvipsnames]{beamer} \fi
\ifnum\conditionmacro=3 \documentclass[notes,handout,usenames,dvipsnames]{beamer}
\usepackage{pgfpages}
\setbeameroption{show notes on second screen=left} \fi

\usetheme{Boadilla}

\input{../HEADERFILES_FOR_LATEX/PACKAGES}
%This file contains all new environments simplifying LaTeX text editing
%For each new environment clearly document its dependencies (packages) and its use
%Adhere to the ordering by category!

%----------------------------------------------------------------------------------------------
%Title Page
%----------------------------------------------------------------------------------------------
\NewDocumentEnvironment{TTitlepage}{mmmm} %Defines an environment for the titlepage
{
\begin{titlepage}
\setlength{\parindent}{0pt}	 \centering
\Huge{\bfseries #1} \\						% X: Document Title
\huge Project TELL \\	[1.cm]

	\begin {table} [H]
		\label {tab:contenttable} \large 
		\centering
		\begin {tabular} {r l} \midrule

		\bfseries{Doc. Reference} & #2 \\ 		% X: Doc. Ref. No.
		\bfseries{Author} & #3  \\						% X: Author Name
		\bfseries{Date} & #4  \\ \\ 						% X: Date of creation

		\end {tabular}
	\end {table}

\begin{figure}[H]
	\centering
   	\includegraphics[width=7cm]{../HEADERFILES_FOR_LATEX/Logo/tell}
\end{figure}

	\vspace*{\fill}
	\begin {table} [H]
		\label {tab:notetable} \large \centering
		\begin {tabular} {l p{12cm}} 

		\midrule
		\bfseries{Note} 
							
		% X: Add notes here if you have any after the &, else write N/A
		&
}{
\end {tabular}
	\end {table}
\end{titlepage}
}
%Usage
%\begin{TTitlepage}{<Document Title>}{<Document Reference Number>}{<Author Name>}{<Date Of Creation>}
%<your notes on the document>
%\end{TTitlepage}
%Dependencies: \usepackage{booktabs,xparse}	
		

%----------------------------------------------------------------------------------------------
%Tables
%----------------------------------------------------------------------------------------------
\NewDocumentEnvironment{TTable}{O{1}mmmm} %Defines a basic tabular environment
{
\begin{table}[H]
\centering
\rowcolors{#1}{white}{gray!25}
\begin{tabular}{#2}
\toprule
}
{
\bottomrule
\end{tabular}
\caption[#3]{#4}
\label{#5}
\end{table}
}
%Usage:
%\begin{TTable}[<startindex for colors>]
%{<table layout>}
%{<your short caption>}
%{<your caption>}
%{<your label>}
%<your Content>
%\end{TTable}
%Dependencies: \usepackage{booktabs,xcolors,xparse}



\NewDocumentEnvironment{TTable*}{mmmm} %Defines a basic tabular environment
{
\begin{table}[H]
\centering
\begin{tabular}{#1}
\toprule
}
{
\bottomrule
\end{tabular}
\caption[#2]{#3}
\label{#4}
\end{table}
}
%Usage:
%\begin{TTable*}{<table layout>}
%{<your short caption>}
%{<your caption>}
%{<your label>}

%<your Content>

%\end{TTable*}



\NewDocumentEnvironment{TSimTable}{mmm} %Defines a tabular environment for simulation data
{
\tiny
\begin{landscape} %rotate the table by 90 degrees
\begin{TTable}[8]{l|l|l|l|l|l|l|l|l|l|l|l|l|l}{#1}{#2}{#3} %Use the TTable environment

%First line of the table head
\multicolumn{2}{l}{}&\multicolumn{7}{l}{\textbf{\large Parameters}}&\multicolumn{5}{l}{\textbf{\large Results}}\\
\cmidrule[2pt]{3-6}\cmidrule[2pt]{10-12}
%Second line of the table head
\multicolumn{2}{l}{}&\multicolumn{3}{l}{Rocket}&\multicolumn{3}{l}{Motor}&\multicolumn{1}{l}{}&\multicolumn{2}{l}{Performance}&\multicolumn{3}{l}{Stability}\\
\cmidrule{3-14}
%Third line of the table head
\mc{l}{SIN}&\mc{l}{Software}&\mc{l}{Name}&\mc{l}{Conf.}&\mc{l}{$m_{tot}$}&\mc{l}{Type}&\mc{l}{$m_{Motor}$}&\mc{l}{$m_{Fuel}$}&\mc{l}{$m_{Stage}$}&\mc{l}{$h$}&\mc{l}{$a_{max}$}&\mc{l}{$v_{LRE}$}&\mc{l}{St. Mar.}&\mc{l}{Dpg. Ratio}\\ %the \mc command gets rid of unneeded vertical lines, see COMMANDS for an explanation
%Forth line of the table head
\multicolumn{4}{l}{}&\mc{l}{$[\si{\kilo\gram}]$}&\mc{l}{}&\mc{l}{$[\si{\kilo\gram}]$}&\mc{l}{$[\si{\kilo\gram}]$}&\mc{l}{$[\si{\kilo\gram}]$}&\mc{l}{$[\si{\kilo\meter}]$}&\mc{l}{$[\si{\meter\per\second\squared}]$}&\mc{l}{$[\si{\meter\per\second}]$}&\multicolumn{2}{l}{}\\
\midrule
}
{
\end{TTable}
\end{landscape}
\normalsize
}
%Usage
%\begin{TSimTable}{<your short caption>}{<your caption>}{<your label>}
%<your content in 14 columns>
%\end{TSimTable}
%Dependencies: \usepackage{lscape,booktabs,siunitx,xcolor,xparse}

\NewDocumentEnvironment{TDefinitionTable}{} %Used to define variables in an equation
{
\footnotesize
\begin{center}
\begin{tabular}{l@{$\quad$}l@{\dotfill$\quad$}cl}
}
{
\end{tabular}
\end{center}
\normalsize
}
%Usage:
%\begin{TDefinitionTable}
%$\vect{P}$&change of linear momentum &in& $\si{\kilo\gram\meter\per\second\squared}$\\
%$m$&mass&in&$\si{\kilo\gram}$\\
%$\vec{a}$&acceleration&in&$\si{\meter\per\second\squared}$\\
%$\vec{F}$&resultant force&in&$\si{N}=\si{\kilo\gram\meter\per\second\squared}$
%\end{TDefinitionTable}

\NewDocumentEnvironment{TValueTable}{} %Used to introduce a series of parameters with certain values
{
\footnotesize
\begin{center}
\begin{tabular}{r@{ = }lr}
}
{
\end{tabular}
\end{center}
\normalsize
}
%Usage:
%\begin{TValueTable}
%$h_E$&$\SI{32}{\milli\meter}$&Extension of the brakes\\
%$h_p$&$\SI{32}{\milli\meter}$&Height of a single plate\\
%$l_B$&$\SI{80}{\milli\meter}$&Width of one brake\\
%$n_p$&$l_B/l_p=\SI{5}{}$&Number of plates\\
%$A_p$&$h_p\cdot l_p\cdot n_p=\SI{25.6}{\centi\meter\squared}$&Total area of the plates\\
%$A_c$&$\SI{26.8}{\centi\meter\squared}$&Actual area of the brake
%\end{TValueTable}


%----------------------------------------------------------------------------------------------
%Appendices
%----------------------------------------------------------------------------------------------
\NewDocumentEnvironment{singlePDFpage}{mmm}
{
\includepdf[pages=-,scale=.75,pagecommand={
\subsection{#1}\label{#2}
},linktodoc=true]
{appendix/#3}
}

\NewDocumentEnvironment{multiPDFpage}{mmm}
{
\includepdf[pages=1,scale=.75,pagecommand={
\subsection{#1}\label{#2}
},linktodoc=true]
{appendix/#3}}
{\includepdf[pages=2-,scale=0.85,pagecommand={},linktodoc=true]{appendix/#3}
}

%This document contains all specially defined environments, used for the presentation template
%The prepended T stands for "Template"
%The prepended TP stands for "TemplatePresentation" - it is the general identifier for environments in this template
%The prepended TF stands for "TemplateFrame" and denotes all environments that create a frame of their own. Multiple commands exist in parallel, as TP and TF, such that they can be combined flexibly.

\usepackage{environ} %Define new environments that work with frames
%Usage for frames:
%\NewEnviron{EnvironmentName}[<#arguments>][<default of optional arguments starting with #1>]{%
%\begin{frame}{<FrameTitle>}
%<macrobeforebody>
%\BODY
%<macroafterbody>
%\end{frame}
%}


\NewEnviron{TFTimeSchedule}[1][]{
\begin{frame}{Time Schedule#1}
\begin{itemize}
\BODY
\end{itemize}
\end{frame}
}
%Usage:
%\begin{TPTimeSchedule}
%\item 15' This
%\item 70' That
%\item 5'  And others
%\end{TPTimeSchedule}

\NewEnviron{TFLearningObjectives}[1][]{
\begin{frame}{Learning Objectives#1}
\begin{itemize}
\BODY
\end{itemize}
\end{frame}
}
%Usage:
%\begin{TPLearningObjectives}
%\item Understand the assumptions for the Euler equations.
%\item Be able to apply the Euler equations to 3D rigid bodies.
%\item Understand the TSP-Rule.
%\end{TPLearningObjectives}

\NewEnviron{TFPicture}[3][]{
\begin{frame}
\begin{figure}
\includegraphics[width=\linewidth,height=0.8\textheight,keepaspectratio]{#1}
\caption{#2}
\label{#3}
\end{figure}
\BODY
\end{frame}
}
%Usage:
%\begin{TFPicture}[Landscape]{An image in landscape orientation.}{fig:Landscape}
%Some text below.
%\end{TFPicture}

\NewEnviron{TPPicture}[3][]{
\begin{figure}
\includegraphics[width=\linewidth,height=0.65\textheight,keepaspectratio]{#1}
\caption{#2}
\label{#3}
\end{figure}
\BODY
}
%Usage:
%\begin{TPPicture}[Landscape]{An image in landscape orientation.}{fig:Landscape}
%Some text below.
%\end{TPPicture}

\NewEnviron{TFTwoColumns}[3][]{
\begin{frame}{#1}
\begin{columns}[T]
\begin{column}{.48\textwidth}
#2
\end{column}%
\hfill%
\begin{column}{.48\textwidth}
#3
\end{column}
\end{columns}
\end{frame}
}
%Usage:
%\begin{TPTwoColumns}[Title]
%{content on the left}
%{and also content on the right}
%\end{TPTwoColumns}

\newcommand{\matlab}{
\lstset{language=Matlab,%
%basicstyle=\color{red},
breaklines=true,%
morekeywords={matlab2tikz},
keywordstyle=\color{blue},%
morekeywords=[2]{1}, keywordstyle=[2]{\color{black}},
identifierstyle=\color{black},%
stringstyle=\color{BlueViolet},
commentstyle=\color{ForestGreen},%
showstringspaces=false,%without this there will be a symbol in the places where there is a space
numbers=left,%
numberstyle={\tiny \color{black}},% size of the numbers
numbersep=9pt, % this defines how far the numbers are from the text
backgroundcolor=\color{yellow!20!white},
frame=single,
escapeinside={\%*}{*)},
}
}
%This file contains all commands simplifying LaTeX text editing
%For each new command clearly document its dependencies (packages) and its use
%Adhere to the ordering by category!

%----------------------------------------------------------------------------------------------
%Maths
%----------------------------------------------------------------------------------------------
\renewcommand{\vec}[1]{\mathbf{#1}} %Notation of vectors
\newcommand{\vect}[1]{\mathbf{\dot{#1}}} %Notation of time derivative of a vector
\newcommand{\vectt}[1]{\mathbf{\ddot{#1}}} %Notation of the second time derivative of a vector
\newcommand{\vecd}[1]{\mathbf{#1'}} %Notation of the first derivative of a vector
\newcommand{\vecdd}[1]{\mathbf{#1''}} 
%In the following all vector definitions will be repeated with the inclusion of a prepended calligraphic letter to define the respective coordinate frame
\newcommand{\cvec}[2]{{}_\mathcal{#1}\mathbf{#2}} %Notation of vectors
\newcommand{\cvect}[2]{{}_\matcal{#1}\mathbf{\dot{#2}}} %Notation of time derivative of a vector
\newcommand{\cvectt}[2]{{}_\matcal{#1}\mathbf{\ddot{#2}}} %Notation of the second time derivative of a vector
\newcommand{\cvecd}[2]{{}_\matcal{#1}\mathbf{#2'}} %Notation of the first derivative of a vector
\newcommand{\cvecdd}[2]{{}_\matcal{#1}\mathbf{#2''}}

%Notation of the second derivative of a vector
\newcommand{\tens}[1]{\underline{\underline{#1}}} %Notation of a tensor
\newcommand{\vprod}[2]{\vec{#1}\times\vec{#2}} %Vector product
\newcommand{\inRone}[1]{\in\mathbb{R}^{#1}} %in 1D set of real numbers
\newcommand{\inRtwo}[2]{\in\mathbb{R}^{#1 \times #2}} %in 2D set of real numbers
\newcommand{\inRthree}[3]{\in\mathbb{R}^{#1 \times #2 \times #3}} %in 3D set of real numbers
\newcommand{\ddt}{\frac{d}{dt}} %time derivative of
\newcommand{\onha}{\frac{1}{2}} %one half
\newcommand{\onth}{\frac{1}{3}} %one third
\newcommand{\onfo}{\frac{1}{4}} %one forth
\newcommand{\prob}[1]{\mathbb{P}(#1)} %Probability of some event
\newcommand{\expe}[1]{\mathbb{E}[#1]} %Expected value for some stochastic variable
\newcommand{\var}[1]{\text{Var}(#1)} %Variance of some stochastic variable
\newcommand{\cov}[1]{\text{Cov}(#1)} %Covariance of some stochastic variables
\newcommand{\corr}[1]{\text{Corr}(#1)} %Correlation of some stochastic variables
\newcommand{\sign}[1]{\text{sign}(#1)} %signum function
\newcommand{\sinc}{\text{sinc}} %sinc function
\newcommand{\diag}{\text{diag}} %diagonal matrix

%----------------------------------------------------------------------------------------------
%Tables
%----------------------------------------------------------------------------------------------

\newcommand{\mc}[2]{\multicolumn{1}{#1}{#2}} %This command is a short for multicolumn
%Usage:
%Instead of \multicolumn{1}{<type>}{<content>}
%write \mc{<type>}{<content>}
%Usecase: When trying to remove vertical lines for only one line in your tabular environment
%multicolumn can come in handy.

%----------------------------------------------------------------------------------------------
% Formating
%----------------------------------------------------------------------------------------------

\newcommand{\sbs}[4]{\begin{minipage}[t!]{#1\linewidth}#3\end{minipage}\begin{minipage}[t!]{#2\linewidth}#4\end{minipage}} %Two minipages dividing the page in two scaled parts
%Usage:
%\sbs{<width of the left, for example 0.45>}{<width on the right, for example 0.45>}
%{
%<content on the left>
%}
%{
%<content on the right>
%}

\newcommand{\sbss}[2]{\sbs{0.45}{0.45}{#1}{#2}} %Two minipages dividing the page in two equal parts
%Usage:
%\sbss
%{
%<content on the left>
%}
%{
%<content on the right>
%}


\newcommand{\sbsss}[3]{\begin{minipage}[t!]{0.42\linewidth}#2\end{minipage}\hspace{#1 ex}\begin{minipage}[t!]{0.42\linewidth}#3\end{minipage}} %Two minipages dividing the page in two equal parts with some user-defined space in [ex] in between
%Usage:
%\sbsss{5}
%{
%<content on the left>
%}
%{
%<content on the right>
%}

%----------------------------------------------------------------------------------------------
%Colors
%----------------------------------------------------------------------------------------------
\definecolor{pantone2128}{RGB}{152,182,228}
\definecolor{pantone2727}{RGB}{48,127,226}
\definecolor{pantone2728}{RGB}{0,71,187}
\definecolor{darkblue}{RGB}{0,35,156}
\definecolor{pantone2757}{RGB}{0,30,96}

%----------------------------------------------------------------------------------------------
%Macros
%----------------------------------------------------------------------------------------------

\newcommand{\statenotation}{
\section{Notation}
\begin{center}
\begin{tabular}{p{0.3\linewidth}p{0.3\linewidth}l}
Operator&Meaning&Example\\
\midrule
\textbf{bold text}&Vector&$\vec{v}$\\
Prepended calligraphic letter&Vector or Matrix represented in the coordinate frame $\mathcal{A}$&$\cvec{A}{v}$\\
Dot above&Time derivative&$\dot{x}$\\
Prime&General derivative&$f'$\\
Double underline&Tensor&$\tens{I}$\\
\midrule
\end{tabular}
\end{center}
}
%Usage:
%Type \statenotation where you'd like to define your notation.

%Checkboxes
\usepackage{pifont}
\newcommand{\cmark}{\ding{51}}%
\newcommand{\xmark}{\ding{55}}%
\newcommand{\done}{\rlap{$\square$}{\raisebox{2pt}{\large\hspace{1pt}\cmark}}%
\hspace{-2.5pt}}
\newcommand{\wontfix}{\rlap{$\square$}{\large\hspace{1pt}\xmark}}
\graphicspath{{./Pictures/}}

\title{Exercise Week 12}
\author{GianAndrea Müller\\ \url{mailto:muellegi@student.ethz}}
\date{\today}

\begin{document}
%Make title page from above supplied information on title, author, etc.
\maketitle

\begin{TFTimeSchedule}
\item 30' Vectors class
\item 10' Dynamisch allokierter Speicherplatz
\item 10' Dynamische Datentypen
\end{TFTimeSchedule}

\begin{TFLearningObjectives}
\item Verständnis: Klassen
\item Kenntnis von Dynamischer Speicherverwaltung
\item Kenntnis von Dynamischen Datenstrukturen
\end{TFLearningObjectives}

\begin{frame}{Klassen: Motivation}
\begin{block}{Was}
\begin{itemize}
\item Klassen: Kombination von Variablen und Funktionen
\item Instanzen
\item Objekte
\end{itemize}
\end{block}

\begin{block}{Objektorientiertes Programmieren}
Die Grundidee besteht darin, die Architektur einer Software an den Grundstrukturen desjenigen Bereichs der Wirklichkeit auszurichten, der die gegebene Anwendung betrifft.
\end{block}

\begin{block}{Wieso}
\begin{itemize}
\item Verkapselung
\item Wiederverwendbarkeit
\end{itemize}
\end{block}

\easyurl{Objektorientiertes Programmieren}{https://de.wikipedia.org/wiki/Objektorientierte_Programmierung}
\end{frame}

\begin{frame}[fragile]{Klassen: Grundlagen}
\begin{block}{Mitglieder}
\begin{itemize}
\item \textbf{Mitgliedsfunktion}:

Funktion die in der Klasse angelegt wird.
\item \textbf{Mitgliedsvariable}: 

Variable die in der Klasse angelegt wird.
\end{itemize}

Mitglieder können nur über eine Instanz der Klasse aufgerufen werden. 
\end{block}

\begin{TPCpp}
//Definition der Klasse
class Vector {...}; 
Vector v1; //Deklaration einer Instanz
v1.memberVariable;
v1.memberFunction();
Vector* pv1 = &v1;
pv1->memberVariable;
\end{TPCpp}
\end{frame}

\note{\begin{itemize}
\item Zugriff auf Mitglieder ueber eine Instanz der
Klasse funktioniert mit dem Punktoperator.
\item Zugriff auf Mitglieder ueber einen Pointer
auf eine Instanz funktioniert mit dem Pfeiloperator.
Dieser ist gleichwertig mit *(pointer).member.
\end{itemize}}

\begin{frame}[fragile]{Klassen: Grundlagen}
\begin{block}{Verkapselung}
\begin{itemize}
\item \textbf{Private Mitglieder}:

Nach dem Schlüsselwort \verb+private+ kommen alle Variablen die versteckt sein sollen.
\item \textbf{Öffentliche Mitglieder}:

Nach dem Schlüsselwort \verb+public+ kommen alle Variablen die öffenlich sein sollen.
\end{itemize}

Nur öffenlichte Mitglieder können über einen der beiden Zugriffsoperatoren erreicht werden. 

\vspace{3ex}

Mitgliedsfunktionen haben Zugriff auf private Migliedsvariablen.
\end{block}
\end{frame}

\begin{frame}[fragile]{Klassen: Grundlagen}
\begin{TFCpp}
class Vector {
private:
	double x;
	double y;
};
\end{TFCpp}

\null\vfill\null

\easyurl{Codeboard}{https://codeboard.io/projects/82028}$\hfill$\easyurl{Zusätzliches Beispiel}{https://codeboard.io/projects/81739}
\end{frame}

\begin{frame}[fragile]{Klassen: Konstruktoren}
\begin{block}{Konstruktor}
\begin{itemize}
\item Öffentliche Mitgliedsfunktion
\item Wird automatisch beim Erstellen einer Instanz der Klasse aufgerufen!
\item Wird eingesetzt um Mitgliedsvariablen zu initialisieren.
\end{itemize}
\end{block}
\begin{TPCpp}
public: 
	Vector (){
		x = 0;
		y = 0;
	}
	Vector (double _x, double _y) : x(_x), y(_y) {}
\end{TPCpp}
\end{frame}

\note{\begin{itemize}
\item Der Konstruktor ist eine Funktion mit dem Namen der Klasse.
\item Der default-Konstruktor, der keine Argumente uebernimmt wird aufgerufen, wenn eine Instanz der Klasse erzeugt wird und keine Argumente uebergeben werden.
\item Der nicht-default-Konstruktor mit 2 Argumenten ist eine Ueberladung des default-Konstruktors.
\item Die Initialisierungsliste wird benutzt um Mitgliedsvariablen 
zu initialisieren. Sie ist hier nicht strikt notwending,
wird aber gebraucht sobald eine Klasse konstante
Mitgliedsvariablen hat, denn diese koennen nur so
initialisiert werden.
\end{itemize}}

\begin{frame}[fragile]{Klassen: Konstruktoren}
\begin{TFCpp}
class Vector {
	double x;
	double y;
public:
	Vector () : x(0),y(0){}
	Vector (double _x, double _y) 
		: x(_x),y(_y) {}
};

Vector v1;
Vector v2();
Vector v3(1.0, 2.3);
\end{TFCpp}
\end{frame}

\note{\begin{itemize}
\item Mitglieder ohne spezifizierung sind automatisch privat!
\item Der Konstruktor wird auch ohne Klammern aufgerufen.
\item Der Konstruktor folgt immer dem Schema
$<$Klassenname$>$ ($<$argumente$>$) $<$initialisierungsliste$>$
$\{<$aktionen$>\}$
\item Der Konstruktor hat immer den Rueckgabetyp void.
Das wird nicht extra angegeben sondern ist so bereits
vorgemerkt.
\end{itemize}}

\begin{frame}[fragile]{Klassen: Zugriffsmethoden}
\begin{block}{Zugriff}
\begin{itemize}
\item Der Zugriff ist durch Verkapselung eingeschränkt.
\item Lösung: Sicherer Zugriff mit Zugriffsmethoden.
\end{itemize}
\end{block}

\vfill

\begin{TPCpp}
double get_x() const {return x;}
double get_y() const {return y;}

void set_x(const double _x) {x = _x;}
void set_y(const double _y) {y = _y;}
\end{TPCpp}
\end{frame}

\note{\begin{itemize}
\item Konstante Mitgliedsfunktionen koennen
Mitgliedsvariablen nicht verändern (ausser diese sind
mutable).
\end{itemize}}

\begin{frame}[fragile]{Klassen: Zugriff auf Mitglieder}
\begin{block}{this}
\begin{itemize}
\item Der \verb+this+ Pointer speichert die Adresse seiner Instanz einer Klasse.
\item Er ist in jeder Klasse vorhanden und kann in Mitgliedsfunktionen benutzt werden um auf die aktuelle Instanz zuzugreifen.
\end{itemize}
\end{block}

\vfill

\begin{TPCpp}
double get_x() const {return this->x;}
double get_y() const {return (*this).y;}
\end{TPCpp}

\end{frame}

\begin{frame}[fragile]{Klassen: Arithmetische Operatoren}
\begin{block}{Argumentübergabe}
\begin{itemize}
\item Die Operatoren, die als Mitgliedsfunktion überladen werden, erhalten die aufrufende Instanz als erstes Argument.
\item Der Rückgabetyp hängt vom überladenen Operator ab. Für Zuweisungen wird eine Referenz zurückgegeben.
\end{itemize}
\end{block}

\begin{TPCpp}
Vector& operator+= (const Vector& b){
    x += b.get_x();
    y += b.get_y();
    return *this;
}
//Im main:
Vector v3(3,4),v4(1,2);
v3 += v4;
\end{TPCpp}
\end{frame}

\begin{frame}[fragile]{Klassen: Arithmetische Operatoren}
\begin{TFCpp}
Vector& operator+= (const Vector& b){
    x += b.get_x();
    y += b.get_y();
    return *this;
}

//Ausserhalb der Klasse
Vector operator+ (const Vector& a, const Vector& b) {
	Vector res = a;
	res += b;
	return res;
}
\end{TFCpp}

\end{frame}

\note{\begin{itemize}
\item Wenn die Ueberladung des += operators ausserhalb
der Klasse stattfindet erlaubt er implizite Typen-
konversion. Ansonten muesste er immer ueber eine
Instanz der Klasse aufgerufen werden.
\end{itemize}}

\note{\begin{itemize}
\item Normalerweise werden diese Funktionen neu
definiert sobald eine dynamische Datenstruktur 
eingefuehrt wird, die nach der Verwaltung von
dynamisch angelegtem Speicherplatz verlangt.
Dann ist es unerlaesslich diese drei Funktionen
entsprechend auszulegen.
\end{itemize}}

\begin{frame}[fragile]{Dynamische allokierter Speicherplatz}
\begin{TFCpp}
int * dyn_int = new int (3);

int size = 5;
int * dyn_array = new int [size];
\end{TFCpp}
\end{frame}

\note{\begin{itemize}
\item Diese beiden Befehle geben einen Zeiger auf
den Anfang des neu allokierten Speicherplatzes zurueck.
\item Die Laenge des neuen Speicherplatzes muss fuer 
Arrays zwingend gespeichert werden!
\item Dynamisch angelegter Speicher muess am Ende der 
Nutzung geloescht werden!
\end{itemize}}

\begin{frame}[fragile]{Dynamische allokierter Speicherplatz}
\begin{TPCpp}
int * dyn_int = new int (3);

int size = 5;
int * dyn_array = new int [size];

delete dyn_int;
dyn_int = 0;
delete[] dyn_array;
dyn_array = NULL;
\end{TPCpp}
\end{frame}

\note{\begin{itemize}
\item Wird der Speicherplatz nicht wieder durch delete
freigegeben ist er fuer das Programm nicht mehr nutzbar.
\item Nach dem Loeschen ist der Speicherplatz
freigegeben, jedoch der Pointer noch vorhanden.
Dieser wird dann auf 0 gesetzt, damit klar ist, dass
er auf nichts zeigt.
\item delete ruft den Destruktor auf.
\end{itemize}}

\begin{frame}[fragile]{Dynamische Datentypen - Stack}
\begin{TFCpp}
class stack {
public:
	void push (int value){...}
	int pop (){}
	...
	void print (){...}
	
private:
	ln* top_node; //ln = list node
};

struct ln {
	int key;
	ln * next;
}
\end{TFCpp}
\end{frame}

\begin{frame}[fragile]{Dynamische Datentypen - Stack}
\begin{TPCpp}
stack s1;
s1.push(1);
s1.push(2);
s1.push(3);
s2(s1);
\end{TPCpp}

\begin{center}
\includegraphics[width=0.9\linewidth]{Pictures/Stack1}
\end{center}

\end{frame}

\note{\begin{itemize}
\item Mit dem Standard Kopierkonstruktor werden einfach
die Mitgliedsvariablen der Struktur kopiert. In
diesem Fall also ein Pointer auf das erste Element
des Stacks.
\item Da wird gerne eine sogenannte tiefe Kopie
haben moechten reicht uns das nicht.
\end{itemize}}

\begin{frame}[fragile]{Dynamische Datentypen - Stack - Tiefe Kopie}
\begin{TFCpp}
stack::stack(const stack& s) : top_node(0) {
	copy(s.top_node, top_node);
}

void stack::copy(const ln* from, ln*& to){
	assert (to == 0);
	if(from != 0){
		to = new ln(from->key);
		copy(from->next, to->next);
	}
}
\end{TFCpp}

\begin{TPCpp}
s2(s1);
\end{TPCpp}

\end{frame}

\note{\begin{itemize}
\item Der Kopierkonstruktor hat als Argument eine Referenz, da die Übergabe eines einfachen Wertes bereits einen Kopiervorgang benötigen würde!
\end{itemize}}

\begin{frame}[fragile]{Dynamische Datentypen - Stack - Zuweisung}
\begin{TPCpp}
stack s2;
s2.push(4);
s2.push(9);
s2 = s1;
\end{TPCpp}


\begin{center}
\includegraphics[width=0.9\linewidth]{Pictures/Stack2}
\end{center}
\end{frame}

\note{\begin{itemize}
\item Der Zuweisungsoperator macht immer noch eine oberflacheliche 
Kopie, setzt also nur alle Mitgliedsvariablen von s1 gleich die von s2.
\item Um das zu beheben ueberladen wir den Zuweisungsoperator.
\end{itemize}}

\begin{frame}[fragile]{Dynamische Datentypen - Stack - Zuweisung}
\begin{TFCpp}
void stack::clear(ln* from){
	if (from != 0){
		clear(from->next);
		delete from;
	}
}
\end{TFCpp}
\end{frame}

\note{\begin{itemize}
\item Bevor zugewiesen werden kann, muss der Empfaenger der Zuweisung
seinen Stack leeren. Dazu wird clear implementiert.
\end{itemize}}

\begin{frame}[fragile]{Dynamische Datenstrukturen - Stack - Zuweisung}
\begin{TPCpp}
stack& stack::operator= (const stack& s) {
	if (top_node != s.top_node) { // test for self-assignment
		clear(top_node);
		top_node = 0; // fix dangling pointer
		copy(s.top_node, top_node);
	}
return *this;
}
\end{TPCpp}

\begin{TPCpp}
s1 = s2;
\end{TPCpp}

\end{frame}

\note{\begin{itemize}
\item Vor der Ueberschreibung wird geprueft ob der 
Stack sich selbst zugewiesen wird.
\item Falls nicht wird die aufrufende Instanz
geloescht und dann mit der Kopierfunktion ueberschrieben
\item Schliesslich wird der Konvention entsprechend
eine Referenz auf die aufrufende Instanz uebergeben.
Das bewirkt weiter, dass der Zuweisungsoperator
verkettet werden kann.
\end{itemize}}

\begin{frame}[fragile]{Dynamische Datenstrukturen - Stack - Destruktor}
\begin{TFCpp}
void useStack(){
	stack temp;
	temp.push(2);
	temp.pop();
} //end of scope, destruction

stack::~stack() {
	clear(top_node);
}
\end{TFCpp}
\end{frame}

\note{\begin{itemize}
\item Beim Loeschen eines Objekts wird immer dessen
Destruktor aufgerufen. Damit im Falle einer
dynamischen Datenstruktur auch alle dynamisch
allokierten Speicherplaetze freigegeben werden
muss der Destruktor entsprechend definiert werden.
\item Daher wird die clear-Funktion fuer alle
angehaengten Nodes aufgerufen. Die uebrigen
Mitglieder der Instanz werden automatisch geloescht.
\end{itemize}}

\begin{frame}{Klassen: Spezielle Mitgliedsfunktionen}
\begin{block}{Standardmitglieder}
\begin{itemize}
\item Defaultkonstruktor
\item Kopierkonstruktor
\item Zuweisungsoperator
\item Defaultdestruktor
\end{itemize}

\end{block}

\textbf{Regel der drei:} Wenn entweder Destruktor, Kopierkonstruktor und Zuweisungsoperator neu definiert werden sollten alle drei neu definiert werden.
\end{frame}

\end{document}

