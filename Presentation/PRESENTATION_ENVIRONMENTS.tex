%This document contains all specially defined environments, used for the presentation template
%The prepended T stands for "Template"
%The prepended TP stands for "TemplatePresentation" - it is the general identifier for environments in this template
%The prepended TF stands for "TemplateFrame" and denotes all environments that create a frame of their own. Multiple commands exist in parallel, as TP and TF, such that they can be combined flexibly.

\usepackage{environ} %Define new environments that work with frames
%Usage for frames:
%\NewEnviron{EnvironmentName}[<#arguments>][<default of optional arguments starting with #1>]{%
%\begin{frame}{<FrameTitle>}
%<macrobeforebody>
%\BODY
%<macroafterbody>
%\end{frame}
%}


\NewEnviron{TFTimeSchedule}[1][]{
\begin{frame}{Time Schedule#1}
\begin{itemize}
\BODY
\end{itemize}
\end{frame}
}
%Usage:
%\begin{TPTimeSchedule}
%\item 15' This
%\item 70' That
%\item 5'  And others
%\end{TPTimeSchedule}

\NewEnviron{TFLearningObjectives}[1][]{
\begin{frame}{Learning Objectives#1}
\begin{itemize}
\BODY
\end{itemize}
\end{frame}
}
%Usage:
%\begin{TPLearningObjectives}
%\item Understand the assumptions for the Euler equations.
%\item Be able to apply the Euler equations to 3D rigid bodies.
%\item Understand the TSP-Rule.
%\end{TPLearningObjectives}

\NewEnviron{TFPicture}[3][]{
\begin{frame}
\begin{figure}
\includegraphics[width=\linewidth,height=0.8\textheight,keepaspectratio]{#1}
\caption{#2}
\label{#3}
\end{figure}
\BODY
\end{frame}
}
%Usage:
%\begin{TFPicture}[Landscape]{An image in landscape orientation.}{fig:Landscape}
%Some text below.
%\end{TFPicture}

\NewEnviron{TPPicture}[3][]{
\begin{figure}
\includegraphics[width=\linewidth,height=0.65\textheight,keepaspectratio]{#1}
\caption{#2}
\label{#3}
\end{figure}
\BODY
}
%Usage:
%\begin{TPPicture}[Landscape]{An image in landscape orientation.}{fig:Landscape}
%Some text below.
%\end{TPPicture}

\NewEnviron{TFTwoColumns}[3][]{
\begin{frame}{#1}
\begin{columns}[T]
\begin{column}{.48\textwidth}
#2
\end{column}%
\hfill%
\begin{column}{.48\textwidth}
#3
\end{column}
\end{columns}
\end{frame}
}
%Usage:
%\begin{TPTwoColumns}[Title]
%{content on the left}
%{and also content on the right}
%\end{TPTwoColumns}

\newcommand{\matlab}{
\lstset{language=Matlab,%
%basicstyle=\color{red},
breaklines=true,%
morekeywords={matlab2tikz},
keywordstyle=\color{blue},%
morekeywords=[2]{1}, keywordstyle=[2]{\color{black}},
identifierstyle=\color{black},%
stringstyle=\color{BlueViolet},
commentstyle=\color{ForestGreen},%
showstringspaces=false,%without this there will be a symbol in the places where there is a space
numbers=left,%
numberstyle={\tiny \color{black}},% size of the numbers
numbersep=9pt, % this defines how far the numbers are from the text
backgroundcolor=\color{yellow!20!white},
frame=single,
escapeinside={\%*}{*)},
}
}