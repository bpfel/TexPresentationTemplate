%This document contains all specially defined environments, used for the presentation template
%The prepended T stands for "Template"
%The prepended TP stands for "TemplatePresentation" - it is the general identifier for environments in this template
%The prepended TF stands for "TemplateFrame" and denotes all environments that create a frame of their own. Multiple commands exist in parallel, as TP and TF, such that they can be combined flexibly.

\usepackage{environ} %Define new environments that work with frames
%Usage for frames:
%\NewEnviron{EnvironmentName}[<#arguments>][<default of optional arguments starting with #1>]{%
%\begin{frame}{<FrameTitle>}
%<macrobeforebody>
%\BODY
%<macroafterbody>
%\end{frame}
%}


\NewEnviron{TFTimeSchedule}[1][]{
\begin{frame}{Time Schedule#1}
\begin{itemize}
\BODY
\end{itemize}
\end{frame}
}
%Usage:
%\begin{TPTimeSchedule}
%\item 15' This
%\item 70' That
%\item 5'  And others
%\end{TPTimeSchedule}

\NewEnviron{TFLearningObjectives}[1][]{
\begin{frame}{Learning Objectives#1}
\begin{itemize}
\BODY
\end{itemize}
\end{frame}
}
%Usage:
%\begin{TPLearningObjectives}
%\item Understand the assumptions for the Euler equations.
%\item Be able to apply the Euler equations to 3D rigid bodies.
%\item Understand the TSP-Rule.
%\end{TPLearningObjectives}

\NewEnviron{TFPicture}[3][]{
\begin{frame}
\begin{figure}
\includegraphics[width=\linewidth,height=0.8\textheight,keepaspectratio]{#1}
\caption{#2}
\label{#3}
\end{figure}
\BODY
\end{frame}
}
%Usage:
%\begin{TFPicture}[Landscape]{An image in landscape orientation.}{fig:Landscape}
%Some text below.
%\end{TFPicture}

\NewEnviron{TPPicture}[3][]{
\begin{figure}
\includegraphics[width=\linewidth,height=0.65\textheight,keepaspectratio]{#1}
\caption{#2}
\label{#3}
\end{figure}
\BODY
}
%Usage:
%\begin{TPPicture}[Landscape]{An image in landscape orientation.}{fig:Landscape}
%Some text below.
%\end{TPPicture}

\NewEnviron{TFTwoColumns}[3][]{
\begin{frame}{#1}
\begin{columns}[T]
\begin{column}{.48\textwidth}
#2
\end{column}%
\hfill%
\begin{column}{.48\textwidth}
#3
\end{column}
\end{columns}
\end{frame}
}
%Usage:
%\begin{TFTwoColumns}[Title]
%{content on the left}
%{and also content on the right}
%\end{TPTwoColumns}

\newcommand{\lsgeneral}{
\lstset{basicstyle=\ttfamily,
breaklines=true,%
showstringspaces=false,%without this there will be a symbol in the places where there is a space
showspaces=false,% show spaces adding particular underscores
showtabs=false, % show tabs within strings adding particular underscores
tabsize=2,
breaklines=true,
numbers=left,%
numberstyle={\tiny \color{black}},% size of the numbers
numbersep=9pt, % this defines how far the numbers are from the text
frame=single,
breakatwhitespace=false,% sets if automatic breaks should only happen at whitespace
xleftmargin=0.05\textwidth,
columns=fixed,
captionpos=b,
}
}

\newcommand{\lsmatlab}{
\lstset{language=Matlab,
keywordstyle=\color{blue},%
keywordstyle=[2]{\color{black}},
identifierstyle=\color{black},%
stringstyle=\color{BlueViolet},
commentstyle=\color{ForestGreen},
backgroundcolor=\color{yellow!20!white},
}
}

\lstnewenvironment{TFMatlab}[1][]{
	\lstset{caption={#1}}
	\lsgeneral
	\lsmatlab
	\centering\minipage{.9\textwidth}%
}{
	\endminipage%
}

\newcommand{\lsbash}{
\lstset{
language=Bash,
backgroundcolor=\color{black!90!white},
keywordstyle=\color{violet!60!white},
commentstyle=\color{gray!60!white},
stringstyle=\color{red!60!white},
basicstyle=\color{white},
morekeywords={nvim, ls}
}
}

\lstnewenvironment{TFBash}[1][]{
	\lstset{caption={#1}}
	\lsgeneral
	\lsbash
	\centering\minipage{.9\textwidth}%
}{
	\endminipage%
}

\newcommand{\lscpp}{
\lstset{ 
	language=C++,% choose the language of the code
	backgroundcolor=\color{black!90!white},
	basicstyle=\color{white}\ttfamily,
    keywordstyle=\color{violet!60!white},
    stringstyle=\color{red!60!white},
    commentstyle=\color{green!60!white},
    morecomment=[l][\color{blue!60!white}]{\#}
}
}

\lstnewenvironment{TFCpp}[1][]{
	\lstset{caption={#1}}
	\lsgeneral
	\lscpp
	\centering\minipage{.9\textwidth}%
}{
	\endminipage%
}

%For multiple fields within one slide
\lstnewenvironment{TPCpp}[1][]{
	\vspace{3ex}
	\lstset{caption={#1}}
	\lsgeneral
	\lscpp
	\centering\minipage{.9\textwidth}%
}{
	\endminipage%
	\vspace{3ex}
}

\newcommand{\lspython}{\lstset{
language=Python,
otherkeywords={self},             % Add keywords here
keywordstyle=\color{black!50!blue},
emph={MyClass,__init__},          % Custom highlighting
emphstyle=\color{black!20!red},    % Custom highlighting style
stringstyle=\color{blue!20!green},
}
}

\lstnewenvironment{TFPython}[1][]{
	\lstset{caption={#1}}
	\lsgeneral
	\lspython
	\centering\minipage{.9\textwidth}%
}{
	\endminipage%
}

\newcommand{\lstex}{\lstset{
language=Tex,
backgroundcolor=\color{brown!8!white},
keywordstyle=\color{black!10!blue},
stringstyle=\color{blue!20!green},
commentstyle=\color{black!40!white},
emph={\section,\subsection,\subsubsection,\input,\begin,\documentclass,\usepackage},          % Custom highlighting
emphstyle=\color{blue}\bfseries,
}
}

\lstnewenvironment{TFLatex}[1][]{
	\lstset{caption={#1}}
	\lsgeneral
	\lstex
	\centering\minipage{.9\textwidth}%
}{
	\endminipage%
}

%Commands

\newcommand{\easyurl}[2]{\href{#2}{\textcolor{blue}{\underline{#1}}}}
