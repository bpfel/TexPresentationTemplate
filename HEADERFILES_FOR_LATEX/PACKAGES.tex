%This file contains all packages and relevant definitions coming with them
%To use it add %This file contains all packages and relevant definitions coming with them
%To use it add %This file contains all packages and relevant definitions coming with them
%To use it add %This file contains all packages and relevant definitions coming with them
%To use it add \input{../HEADERFILES_FOR_LATEX/PACKAGES} to the head of your file

%----------------------------------------------------------------------------------------------
%Typesetting
%----------------------------------------------------------------------------------------------
\usepackage[utf8]{inputenc} %To allow typesetting öäü

%----------------------------------------------------------------------------------------------
%Typesetting code
%----------------------------------------------------------------------------------------------
\usepackage{listings} %To allow nice typesetting of code

%Usage:
%Definition of all parameters defining the appearance of your code
%\lstset{ 
%	language=Matlab,% choose the language of the code
%	basicstyle=10pt,% the size of the fonts that are used for the code
%	numbers=left,% where to put the line-numbers
%	numberstyle=\footnotesize,% the size of the fonts that are used for the line-numbers
%	stepnumber=1,% the step between two line-numbers. If it's 1 each line will be numbered
%	numbersep=5pt,% how far the line-numbers are from the code
%	backgroundcolor=\color{SilverP},% choose the background color. You must add \usepackage{color}
%	showspaces=false,% show spaces adding particular underscores
%	showstringspaces=false,% underline spaces within strings
%	showtabs=false,% show tabs within strings adding particular underscores
% 	frame=single,% adds a frame around the code
% 	tabsize=2,% sets default tabsize to 2 spaces
% 	captionpos=t,% sets the caption-position to top
%	breaklines=true,% sets automatic line breaking
%	breakatwhitespace=false,% sets if automatic breaks should only happen at whitespace
%	escapeinside={\%*}{*)}% if you want to add a comment within your code
%}
%How to use it inside your document:
%\lstlistoflistings % To list all listings
%
%To input a whole file
%\lstinputlisting[language=Matlab,caption = Main file, label = lst:MainFile]{Code/LDC.m}
%
%To input some lines of code
%\begin{lstlisting
%#include <iostream>
%\end{lstlisting}


%----------------------------------------------------------------------------------------------
%Maths
%----------------------------------------------------------------------------------------------
\usepackage{amsmath} %Basic math package
\usepackage{amssymb} %Special symbols
\usepackage{siunitx} %Nice typesetting of numbers with units
%Usage:
%Use in math mode!
%For a unit without a number: \si{\kilo\gram\per\second}
%For a number with a unit: \SI{10e-12}{\nano\meter\per\second\squared}

%----------------------------------------------------------------------------------------------
%Formating
%----------------------------------------------------------------------------------------------
\usepackage{graphicx} %Placing of pictures
%Usage:
%\includegraphics[width=\linewidth]{<Name of picture in pictures without filetype extension/>}
\graphicspath{ {pictures/} } %Allows to include pictures with only their name instead of the path

\usepackage{multicol} %Dividing the page into multiple columns
%Usage:
%\begin{multicols}{<number of colums>}<your content>\end{multicols}

\usepackage{wrapfig} %Allows to wrap text around figures
%Usage:
%\begin{wrapfigure}[<Number of narrow lines>]{<placement>}[<overhang in pt> pt]
% <some figure>
%\end{wrapfigure}

%\usepackage[table]{xcolor} %Allows the use of an array of colors
%Option: table Load the colortbl package, in order to use the tools for coloring rows, columns, and cells within tables
%Usage:
%Change textcolor: \textcolor{<color>}{<text>}
%Make a colorbox: \fcolorbox{<box color>{<background color>} 

\usepackage{booktabs} %Allows making nicer tables
%Usage:
%Make a nice horizontal midrule: \midrule

\usepackage{float} %An improved environment for floating figures

\usepackage{fancyhdr} %Allows manipulation of headers and footers

\usepackage{blindtext} %Allows generation of blind text
%Usage:
%One paragraph of lorem ipsum: \blindtext
%A certain number of paragraphs: \Blindtext[X][Y]
%where X is the number of paragraphs and Y is the number of pangrams

\usepackage{enumitem} %Allows configure enumerations and such
%Usage:
%\begin{itemize}[leftmargin=*,noitemsep,topsep=0pt]
%\item <content>
%\end{itemize}
%This examples leaves away the margin on the left, sets the item seperation to zero and leaves away the
%white space generated above an enumeration/itemization

\usepackage{longtable, tabularx} %Allows the introduction of special columntypes for tabular environments
\newcolumntype{L}[1]{>{\raggedright\arraybackslash}p{#1}} % raggedleft column 
\newcolumntype{C}[1]{>{\centering\arraybackslash}p{#1}} % centered column
\newcolumntype{R}[1]{>{\raggedleft\arraybackslash}p{#1}} % raggedright column
%Usage:
%\begin{tabular}{L{2cm}|L{2cm}|L{2cm}|L{3cm}|L{3cm}}
%<content>
%\end{tabular}

\usepackage{lscape} %Allows the use of single landscape pages within a document in portrait orientation
%Usage:
%\begin{landscape}
%<content>
%\end{landscape}

%\usepackage[hidelinks]{hyperref} %allows for clickable references, doesn't colour the links

\usepackage{pgfplots} %allows including specially exported .svg-figures
%You want nice vector-graphics in your document that even have latex-fonts? Ask Gianni.
%Also allows doing tikzpictures!
%Usage:
%\begin{tikzpicture}
%\begin{axis}[
%    axis lines = left,
%    xlabel = {$h$ in $\si{mm}$},
%    ylabel = {$C_{D\Theta,corrected}$},
%]
%\addplot [
%    domain=5:45, 
%    samples=100, 
%    color=MyPomegranate,
%]
%{1.17*(1-0.25*8.93/x)};
%\end{axis}
%\end{tikzpicture}}


%\usepackage[ %Allows importing your bibliography from another file
%backend = biber,
%style = ieee,
%natbib = true,
%url = true,
%doi = false,
%eprint = false,
%]{biblatex}
%Usage:
%\addbibresource{<Absolute/Path/To/Your/.bib/file.bib>}
%\printbiblbiography %Add this line where you'd like to print your bibliography

%----------------------------------------------------------------------------------------------
%LaTeX workarounds
%----------------------------------------------------------------------------------------------

\usepackage{xparse} %Allows to define new environments with arguments passed to the \end{env}-part
%Consider reading: https://tex.stackexchange.com/questions/17036/why-cant-the-end-code-of-an-environment-contain-an-argument
%Usage:
%\NewDocumentEnvironment{foo}{m}
%  {some code #1}
%  {some code #1}

%----------------------------------------------------------------------------------------------
%Additions AIAA
%----------------------------------------------------------------------------------------------
\usepackage[version=4]{mhchem} %Allows using signs for chemical equations and other fancy stuff
\usepackage{pdfpages} %To add pdf files
\usepackage{appendix} %For appendix typesettings to the head of your file

%----------------------------------------------------------------------------------------------
%Typesetting
%----------------------------------------------------------------------------------------------
\usepackage[utf8]{inputenc} %To allow typesetting öäü

%----------------------------------------------------------------------------------------------
%Typesetting code
%----------------------------------------------------------------------------------------------
\usepackage{listings} %To allow nice typesetting of code

%Usage:
%Definition of all parameters defining the appearance of your code
%\lstset{ 
%	language=Matlab,% choose the language of the code
%	basicstyle=10pt,% the size of the fonts that are used for the code
%	numbers=left,% where to put the line-numbers
%	numberstyle=\footnotesize,% the size of the fonts that are used for the line-numbers
%	stepnumber=1,% the step between two line-numbers. If it's 1 each line will be numbered
%	numbersep=5pt,% how far the line-numbers are from the code
%	backgroundcolor=\color{SilverP},% choose the background color. You must add \usepackage{color}
%	showspaces=false,% show spaces adding particular underscores
%	showstringspaces=false,% underline spaces within strings
%	showtabs=false,% show tabs within strings adding particular underscores
% 	frame=single,% adds a frame around the code
% 	tabsize=2,% sets default tabsize to 2 spaces
% 	captionpos=t,% sets the caption-position to top
%	breaklines=true,% sets automatic line breaking
%	breakatwhitespace=false,% sets if automatic breaks should only happen at whitespace
%	escapeinside={\%*}{*)}% if you want to add a comment within your code
%}
%How to use it inside your document:
%\lstlistoflistings % To list all listings
%
%To input a whole file
%\lstinputlisting[language=Matlab,caption = Main file, label = lst:MainFile]{Code/LDC.m}
%
%To input some lines of code
%\begin{lstlisting
%#include <iostream>
%\end{lstlisting}


%----------------------------------------------------------------------------------------------
%Maths
%----------------------------------------------------------------------------------------------
\usepackage{amsmath} %Basic math package
\usepackage{amssymb} %Special symbols
\usepackage{siunitx} %Nice typesetting of numbers with units
%Usage:
%Use in math mode!
%For a unit without a number: \si{\kilo\gram\per\second}
%For a number with a unit: \SI{10e-12}{\nano\meter\per\second\squared}

%----------------------------------------------------------------------------------------------
%Formating
%----------------------------------------------------------------------------------------------
\usepackage{graphicx} %Placing of pictures
%Usage:
%\includegraphics[width=\linewidth]{<Name of picture in pictures without filetype extension/>}
\graphicspath{ {pictures/} } %Allows to include pictures with only their name instead of the path

\usepackage{multicol} %Dividing the page into multiple columns
%Usage:
%\begin{multicols}{<number of colums>}<your content>\end{multicols}

\usepackage{wrapfig} %Allows to wrap text around figures
%Usage:
%\begin{wrapfigure}[<Number of narrow lines>]{<placement>}[<overhang in pt> pt]
% <some figure>
%\end{wrapfigure}

%\usepackage[table]{xcolor} %Allows the use of an array of colors
%Option: table Load the colortbl package, in order to use the tools for coloring rows, columns, and cells within tables
%Usage:
%Change textcolor: \textcolor{<color>}{<text>}
%Make a colorbox: \fcolorbox{<box color>{<background color>} 

\usepackage{booktabs} %Allows making nicer tables
%Usage:
%Make a nice horizontal midrule: \midrule

\usepackage{float} %An improved environment for floating figures

\usepackage{fancyhdr} %Allows manipulation of headers and footers

\usepackage{blindtext} %Allows generation of blind text
%Usage:
%One paragraph of lorem ipsum: \blindtext
%A certain number of paragraphs: \Blindtext[X][Y]
%where X is the number of paragraphs and Y is the number of pangrams

\usepackage{enumitem} %Allows configure enumerations and such
%Usage:
%\begin{itemize}[leftmargin=*,noitemsep,topsep=0pt]
%\item <content>
%\end{itemize}
%This examples leaves away the margin on the left, sets the item seperation to zero and leaves away the
%white space generated above an enumeration/itemization

\usepackage{longtable, tabularx} %Allows the introduction of special columntypes for tabular environments
\newcolumntype{L}[1]{>{\raggedright\arraybackslash}p{#1}} % raggedleft column 
\newcolumntype{C}[1]{>{\centering\arraybackslash}p{#1}} % centered column
\newcolumntype{R}[1]{>{\raggedleft\arraybackslash}p{#1}} % raggedright column
%Usage:
%\begin{tabular}{L{2cm}|L{2cm}|L{2cm}|L{3cm}|L{3cm}}
%<content>
%\end{tabular}

\usepackage{lscape} %Allows the use of single landscape pages within a document in portrait orientation
%Usage:
%\begin{landscape}
%<content>
%\end{landscape}

%\usepackage[hidelinks]{hyperref} %allows for clickable references, doesn't colour the links

\usepackage{pgfplots} %allows including specially exported .svg-figures
%You want nice vector-graphics in your document that even have latex-fonts? Ask Gianni.
%Also allows doing tikzpictures!
%Usage:
%\begin{tikzpicture}
%\begin{axis}[
%    axis lines = left,
%    xlabel = {$h$ in $\si{mm}$},
%    ylabel = {$C_{D\Theta,corrected}$},
%]
%\addplot [
%    domain=5:45, 
%    samples=100, 
%    color=MyPomegranate,
%]
%{1.17*(1-0.25*8.93/x)};
%\end{axis}
%\end{tikzpicture}}


%\usepackage[ %Allows importing your bibliography from another file
%backend = biber,
%style = ieee,
%natbib = true,
%url = true,
%doi = false,
%eprint = false,
%]{biblatex}
%Usage:
%\addbibresource{<Absolute/Path/To/Your/.bib/file.bib>}
%\printbiblbiography %Add this line where you'd like to print your bibliography

%----------------------------------------------------------------------------------------------
%LaTeX workarounds
%----------------------------------------------------------------------------------------------

\usepackage{xparse} %Allows to define new environments with arguments passed to the \end{env}-part
%Consider reading: https://tex.stackexchange.com/questions/17036/why-cant-the-end-code-of-an-environment-contain-an-argument
%Usage:
%\NewDocumentEnvironment{foo}{m}
%  {some code #1}
%  {some code #1}

%----------------------------------------------------------------------------------------------
%Additions AIAA
%----------------------------------------------------------------------------------------------
\usepackage[version=4]{mhchem} %Allows using signs for chemical equations and other fancy stuff
\usepackage{pdfpages} %To add pdf files
\usepackage{appendix} %For appendix typesettings to the head of your file

%----------------------------------------------------------------------------------------------
%Typesetting
%----------------------------------------------------------------------------------------------
\usepackage[utf8]{inputenc} %To allow typesetting öäü

%----------------------------------------------------------------------------------------------
%Typesetting code
%----------------------------------------------------------------------------------------------
\usepackage{listings} %To allow nice typesetting of code

%Usage:
%Definition of all parameters defining the appearance of your code
%\lstset{ 
%	language=Matlab,% choose the language of the code
%	basicstyle=10pt,% the size of the fonts that are used for the code
%	numbers=left,% where to put the line-numbers
%	numberstyle=\footnotesize,% the size of the fonts that are used for the line-numbers
%	stepnumber=1,% the step between two line-numbers. If it's 1 each line will be numbered
%	numbersep=5pt,% how far the line-numbers are from the code
%	backgroundcolor=\color{SilverP},% choose the background color. You must add \usepackage{color}
%	showspaces=false,% show spaces adding particular underscores
%	showstringspaces=false,% underline spaces within strings
%	showtabs=false,% show tabs within strings adding particular underscores
% 	frame=single,% adds a frame around the code
% 	tabsize=2,% sets default tabsize to 2 spaces
% 	captionpos=t,% sets the caption-position to top
%	breaklines=true,% sets automatic line breaking
%	breakatwhitespace=false,% sets if automatic breaks should only happen at whitespace
%	escapeinside={\%*}{*)}% if you want to add a comment within your code
%}
%How to use it inside your document:
%\lstlistoflistings % To list all listings
%
%To input a whole file
%\lstinputlisting[language=Matlab,caption = Main file, label = lst:MainFile]{Code/LDC.m}
%
%To input some lines of code
%\begin{lstlisting
%#include <iostream>
%\end{lstlisting}


%----------------------------------------------------------------------------------------------
%Maths
%----------------------------------------------------------------------------------------------
\usepackage{amsmath} %Basic math package
\usepackage{amssymb} %Special symbols
\usepackage{siunitx} %Nice typesetting of numbers with units
%Usage:
%Use in math mode!
%For a unit without a number: \si{\kilo\gram\per\second}
%For a number with a unit: \SI{10e-12}{\nano\meter\per\second\squared}

%----------------------------------------------------------------------------------------------
%Formating
%----------------------------------------------------------------------------------------------
\usepackage{graphicx} %Placing of pictures
%Usage:
%\includegraphics[width=\linewidth]{<Name of picture in pictures without filetype extension/>}
\graphicspath{ {pictures/} } %Allows to include pictures with only their name instead of the path

\usepackage{multicol} %Dividing the page into multiple columns
%Usage:
%\begin{multicols}{<number of colums>}<your content>\end{multicols}

\usepackage{wrapfig} %Allows to wrap text around figures
%Usage:
%\begin{wrapfigure}[<Number of narrow lines>]{<placement>}[<overhang in pt> pt]
% <some figure>
%\end{wrapfigure}

%\usepackage[table]{xcolor} %Allows the use of an array of colors
%Option: table Load the colortbl package, in order to use the tools for coloring rows, columns, and cells within tables
%Usage:
%Change textcolor: \textcolor{<color>}{<text>}
%Make a colorbox: \fcolorbox{<box color>{<background color>} 

\usepackage{booktabs} %Allows making nicer tables
%Usage:
%Make a nice horizontal midrule: \midrule

\usepackage{float} %An improved environment for floating figures

\usepackage{fancyhdr} %Allows manipulation of headers and footers

\usepackage{blindtext} %Allows generation of blind text
%Usage:
%One paragraph of lorem ipsum: \blindtext
%A certain number of paragraphs: \Blindtext[X][Y]
%where X is the number of paragraphs and Y is the number of pangrams

\usepackage{enumitem} %Allows configure enumerations and such
%Usage:
%\begin{itemize}[leftmargin=*,noitemsep,topsep=0pt]
%\item <content>
%\end{itemize}
%This examples leaves away the margin on the left, sets the item seperation to zero and leaves away the
%white space generated above an enumeration/itemization

\usepackage{longtable, tabularx} %Allows the introduction of special columntypes for tabular environments
\newcolumntype{L}[1]{>{\raggedright\arraybackslash}p{#1}} % raggedleft column 
\newcolumntype{C}[1]{>{\centering\arraybackslash}p{#1}} % centered column
\newcolumntype{R}[1]{>{\raggedleft\arraybackslash}p{#1}} % raggedright column
%Usage:
%\begin{tabular}{L{2cm}|L{2cm}|L{2cm}|L{3cm}|L{3cm}}
%<content>
%\end{tabular}

\usepackage{lscape} %Allows the use of single landscape pages within a document in portrait orientation
%Usage:
%\begin{landscape}
%<content>
%\end{landscape}

%\usepackage[hidelinks]{hyperref} %allows for clickable references, doesn't colour the links

\usepackage{pgfplots} %allows including specially exported .svg-figures
%You want nice vector-graphics in your document that even have latex-fonts? Ask Gianni.
%Also allows doing tikzpictures!
%Usage:
%\begin{tikzpicture}
%\begin{axis}[
%    axis lines = left,
%    xlabel = {$h$ in $\si{mm}$},
%    ylabel = {$C_{D\Theta,corrected}$},
%]
%\addplot [
%    domain=5:45, 
%    samples=100, 
%    color=MyPomegranate,
%]
%{1.17*(1-0.25*8.93/x)};
%\end{axis}
%\end{tikzpicture}}


%\usepackage[ %Allows importing your bibliography from another file
%backend = biber,
%style = ieee,
%natbib = true,
%url = true,
%doi = false,
%eprint = false,
%]{biblatex}
%Usage:
%\addbibresource{<Absolute/Path/To/Your/.bib/file.bib>}
%\printbiblbiography %Add this line where you'd like to print your bibliography

%----------------------------------------------------------------------------------------------
%LaTeX workarounds
%----------------------------------------------------------------------------------------------

\usepackage{xparse} %Allows to define new environments with arguments passed to the \end{env}-part
%Consider reading: https://tex.stackexchange.com/questions/17036/why-cant-the-end-code-of-an-environment-contain-an-argument
%Usage:
%\NewDocumentEnvironment{foo}{m}
%  {some code #1}
%  {some code #1}

%----------------------------------------------------------------------------------------------
%Additions AIAA
%----------------------------------------------------------------------------------------------
\usepackage[version=4]{mhchem} %Allows using signs for chemical equations and other fancy stuff
\usepackage{pdfpages} %To add pdf files
\usepackage{appendix} %For appendix typesettings to the head of your file

%----------------------------------------------------------------------------------------------
%Typesetting
%----------------------------------------------------------------------------------------------
\usepackage[utf8]{inputenc} %To allow typesetting öäü

%----------------------------------------------------------------------------------------------
%Typesetting code
%----------------------------------------------------------------------------------------------
\usepackage{listings} %To allow nice typesetting of code

%Usage:
%Definition of all parameters defining the appearance of your code
%\lstset{ 
%	language=Matlab,% choose the language of the code
%	basicstyle=10pt,% the size of the fonts that are used for the code
%	numbers=left,% where to put the line-numbers
%	numberstyle=\footnotesize,% the size of the fonts that are used for the line-numbers
%	stepnumber=1,% the step between two line-numbers. If it's 1 each line will be numbered
%	numbersep=5pt,% how far the line-numbers are from the code
%	backgroundcolor=\color{SilverP},% choose the background color. You must add \usepackage{color}
%	showspaces=false,% show spaces adding particular underscores
%	showstringspaces=false,% underline spaces within strings
%	showtabs=false,% show tabs within strings adding particular underscores
% 	frame=single,% adds a frame around the code
% 	tabsize=2,% sets default tabsize to 2 spaces
% 	captionpos=t,% sets the caption-position to top
%	breaklines=true,% sets automatic line breaking
%	breakatwhitespace=false,% sets if automatic breaks should only happen at whitespace
%	escapeinside={\%*}{*)}% if you want to add a comment within your code
%}
%How to use it inside your document:
%\lstlistoflistings % To list all listings
%
%To input a whole file
%\lstinputlisting[language=Matlab,caption = Main file, label = lst:MainFile]{Code/LDC.m}
%
%To input some lines of code
%\begin{lstlisting
%#include <iostream>
%\end{lstlisting}


%----------------------------------------------------------------------------------------------
%Maths
%----------------------------------------------------------------------------------------------
\usepackage{amsmath} %Basic math package
\usepackage{amssymb} %Special symbols
\usepackage{siunitx} %Nice typesetting of numbers with units
%Usage:
%Use in math mode!
%For a unit without a number: \si{\kilo\gram\per\second}
%For a number with a unit: \SI{10e-12}{\nano\meter\per\second\squared}

%----------------------------------------------------------------------------------------------
%Formating
%----------------------------------------------------------------------------------------------
\usepackage{graphicx} %Placing of pictures
%Usage:
%\includegraphics[width=\linewidth]{<Name of picture in pictures without filetype extension/>}
\graphicspath{ {pictures/} } %Allows to include pictures with only their name instead of the path

\usepackage{multicol} %Dividing the page into multiple columns
%Usage:
%\begin{multicols}{<number of colums>}<your content>\end{multicols}

\usepackage{wrapfig} %Allows to wrap text around figures
%Usage:
%\begin{wrapfigure}[<Number of narrow lines>]{<placement>}[<overhang in pt> pt]
% <some figure>
%\end{wrapfigure}

%\usepackage[table]{xcolor} %Allows the use of an array of colors
%Option: table Load the colortbl package, in order to use the tools for coloring rows, columns, and cells within tables
%Usage:
%Change textcolor: \textcolor{<color>}{<text>}
%Make a colorbox: \fcolorbox{<box color>{<background color>} 

\usepackage{booktabs} %Allows making nicer tables
%Usage:
%Make a nice horizontal midrule: \midrule

\usepackage{float} %An improved environment for floating figures

\usepackage{fancyhdr} %Allows manipulation of headers and footers

\usepackage{blindtext} %Allows generation of blind text
%Usage:
%One paragraph of lorem ipsum: \blindtext
%A certain number of paragraphs: \Blindtext[X][Y]
%where X is the number of paragraphs and Y is the number of pangrams

\usepackage{enumitem} %Allows configure enumerations and such
%Usage:
%\begin{itemize}[leftmargin=*,noitemsep,topsep=0pt]
%\item <content>
%\end{itemize}
%This examples leaves away the margin on the left, sets the item seperation to zero and leaves away the
%white space generated above an enumeration/itemization

\usepackage{longtable, tabularx} %Allows the introduction of special columntypes for tabular environments
\newcolumntype{L}[1]{>{\raggedright\arraybackslash}p{#1}} % raggedleft column 
\newcolumntype{C}[1]{>{\centering\arraybackslash}p{#1}} % centered column
\newcolumntype{R}[1]{>{\raggedleft\arraybackslash}p{#1}} % raggedright column
%Usage:
%\begin{tabular}{L{2cm}|L{2cm}|L{2cm}|L{3cm}|L{3cm}}
%<content>
%\end{tabular}

\usepackage{lscape} %Allows the use of single landscape pages within a document in portrait orientation
%Usage:
%\begin{landscape}
%<content>
%\end{landscape}

%\usepackage[hidelinks]{hyperref} %allows for clickable references, doesn't colour the links

\usepackage{pgfplots} %allows including specially exported .svg-figures
%You want nice vector-graphics in your document that even have latex-fonts? Ask Gianni.
%Also allows doing tikzpictures!
%Usage:
%\begin{tikzpicture}
%\begin{axis}[
%    axis lines = left,
%    xlabel = {$h$ in $\si{mm}$},
%    ylabel = {$C_{D\Theta,corrected}$},
%]
%\addplot [
%    domain=5:45, 
%    samples=100, 
%    color=MyPomegranate,
%]
%{1.17*(1-0.25*8.93/x)};
%\end{axis}
%\end{tikzpicture}}


%\usepackage[ %Allows importing your bibliography from another file
%backend = biber,
%style = ieee,
%natbib = true,
%url = true,
%doi = false,
%eprint = false,
%]{biblatex}
%Usage:
%\addbibresource{<Absolute/Path/To/Your/.bib/file.bib>}
%\printbiblbiography %Add this line where you'd like to print your bibliography

%----------------------------------------------------------------------------------------------
%LaTeX workarounds
%----------------------------------------------------------------------------------------------

\usepackage{xparse} %Allows to define new environments with arguments passed to the \end{env}-part
%Consider reading: https://tex.stackexchange.com/questions/17036/why-cant-the-end-code-of-an-environment-contain-an-argument
%Usage:
%\NewDocumentEnvironment{foo}{m}
%  {some code #1}
%  {some code #1}

%----------------------------------------------------------------------------------------------
%Additions AIAA
%----------------------------------------------------------------------------------------------
\usepackage[version=4]{mhchem} %Allows using signs for chemical equations and other fancy stuff
\usepackage{pdfpages} %To add pdf files
\usepackage{appendix} %For appendix typesettings