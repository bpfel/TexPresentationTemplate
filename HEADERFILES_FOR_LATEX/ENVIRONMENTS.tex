%This file contains all new environments simplifying LaTeX text editing
%For each new environment clearly document its dependencies (packages) and its use
%Adhere to the ordering by category!

%----------------------------------------------------------------------------------------------
%Title Page
%----------------------------------------------------------------------------------------------
\NewDocumentEnvironment{TTitlepage}{mmmm} %Defines an environment for the titlepage
{
\begin{titlepage}
\setlength{\parindent}{0pt}	 \centering
\Huge{\bfseries #1} \\						% X: Document Title
\huge Project TELL \\	[1.cm]

	\begin {table} [H]
		\label {tab:contenttable} \large 
		\centering
		\begin {tabular} {r l} \midrule

		\bfseries{Doc. Reference} & #2 \\ 		% X: Doc. Ref. No.
		\bfseries{Author} & #3  \\						% X: Author Name
		\bfseries{Date} & #4  \\ \\ 						% X: Date of creation

		\end {tabular}
	\end {table}

\begin{figure}[H]
	\centering
   	\includegraphics[width=7cm]{../HEADERFILES_FOR_LATEX/Logo/tell}
\end{figure}

	\vspace*{\fill}
	\begin {table} [H]
		\label {tab:notetable} \large \centering
		\begin {tabular} {l p{12cm}} 

		\midrule
		\bfseries{Note} 
							
		% X: Add notes here if you have any after the &, else write N/A
		&
}{
\end {tabular}
	\end {table}
\end{titlepage}
}
%Usage
%\begin{TTitlepage}{<Document Title>}{<Document Reference Number>}{<Author Name>}{<Date Of Creation>}
%<your notes on the document>
%\end{TTitlepage}
%Dependencies: \usepackage{booktabs,xparse}	
		

%----------------------------------------------------------------------------------------------
%Tables
%----------------------------------------------------------------------------------------------
\NewDocumentEnvironment{TTable}{O{1}mmmm} %Defines a basic tabular environment
{
\begin{table}[H]
\centering
\rowcolors{#1}{white}{gray!25}
\begin{tabular}{#2}
\toprule
}
{
\bottomrule
\end{tabular}
\caption[#3]{#4}
\label{#5}
\end{table}
}
%Usage:
%\begin{TTable}[<startindex for colors>]
%{<table layout>}
%{<your short caption>}
%{<your caption>}
%{<your label>}
%<your Content>
%\end{TTable}
%Dependencies: \usepackage{booktabs,xcolors,xparse}



\NewDocumentEnvironment{TTable*}{mmmm} %Defines a basic tabular environment
{
\begin{table}[H]
\centering
\begin{tabular}{#1}
\toprule
}
{
\bottomrule
\end{tabular}
\caption[#2]{#3}
\label{#4}
\end{table}
}
%Usage:
%\begin{TTable*}{<table layout>}
%{<your short caption>}
%{<your caption>}
%{<your label>}

%<your Content>

%\end{TTable*}



\NewDocumentEnvironment{TSimTable}{mmm} %Defines a tabular environment for simulation data
{
\tiny
\begin{landscape} %rotate the table by 90 degrees
\begin{TTable}[8]{l|l|l|l|l|l|l|l|l|l|l|l|l|l}{#1}{#2}{#3} %Use the TTable environment

%First line of the table head
\multicolumn{2}{l}{}&\multicolumn{7}{l}{\textbf{\large Parameters}}&\multicolumn{5}{l}{\textbf{\large Results}}\\
\cmidrule[2pt]{3-6}\cmidrule[2pt]{10-12}
%Second line of the table head
\multicolumn{2}{l}{}&\multicolumn{3}{l}{Rocket}&\multicolumn{3}{l}{Motor}&\multicolumn{1}{l}{}&\multicolumn{2}{l}{Performance}&\multicolumn{3}{l}{Stability}\\
\cmidrule{3-14}
%Third line of the table head
\mc{l}{SIN}&\mc{l}{Software}&\mc{l}{Name}&\mc{l}{Conf.}&\mc{l}{$m_{tot}$}&\mc{l}{Type}&\mc{l}{$m_{Motor}$}&\mc{l}{$m_{Fuel}$}&\mc{l}{$m_{Stage}$}&\mc{l}{$h$}&\mc{l}{$a_{max}$}&\mc{l}{$v_{LRE}$}&\mc{l}{St. Mar.}&\mc{l}{Dpg. Ratio}\\ %the \mc command gets rid of unneeded vertical lines, see COMMANDS for an explanation
%Forth line of the table head
\multicolumn{4}{l}{}&\mc{l}{$[\si{\kilo\gram}]$}&\mc{l}{}&\mc{l}{$[\si{\kilo\gram}]$}&\mc{l}{$[\si{\kilo\gram}]$}&\mc{l}{$[\si{\kilo\gram}]$}&\mc{l}{$[\si{\kilo\meter}]$}&\mc{l}{$[\si{\meter\per\second\squared}]$}&\mc{l}{$[\si{\meter\per\second}]$}&\multicolumn{2}{l}{}\\
\midrule
}
{
\end{TTable}
\end{landscape}
\normalsize
}
%Usage
%\begin{TSimTable}{<your short caption>}{<your caption>}{<your label>}
%<your content in 14 columns>
%\end{TSimTable}
%Dependencies: \usepackage{lscape,booktabs,siunitx,xcolor,xparse}

\NewDocumentEnvironment{TDefinitionTable}{} %Used to define variables in an equation
{
\footnotesize
\begin{center}
\begin{tabular}{l@{$\quad$}l@{\dotfill$\quad$}cl}
}
{
\end{tabular}
\end{center}
\normalsize
}
%Usage:
%\begin{TDefinitionTable}
%$\vect{P}$&change of linear momentum &in& $\si{\kilo\gram\meter\per\second\squared}$\\
%$m$&mass&in&$\si{\kilo\gram}$\\
%$\vec{a}$&acceleration&in&$\si{\meter\per\second\squared}$\\
%$\vec{F}$&resultant force&in&$\si{N}=\si{\kilo\gram\meter\per\second\squared}$
%\end{TDefinitionTable}

\NewDocumentEnvironment{TValueTable}{} %Used to introduce a series of parameters with certain values
{
\footnotesize
\begin{center}
\begin{tabular}{r@{ = }lr}
}
{
\end{tabular}
\end{center}
\normalsize
}
%Usage:
%\begin{TValueTable}
%$h_E$&$\SI{32}{\milli\meter}$&Extension of the brakes\\
%$h_p$&$\SI{32}{\milli\meter}$&Height of a single plate\\
%$l_B$&$\SI{80}{\milli\meter}$&Width of one brake\\
%$n_p$&$l_B/l_p=\SI{5}{}$&Number of plates\\
%$A_p$&$h_p\cdot l_p\cdot n_p=\SI{25.6}{\centi\meter\squared}$&Total area of the plates\\
%$A_c$&$\SI{26.8}{\centi\meter\squared}$&Actual area of the brake
%\end{TValueTable}


%----------------------------------------------------------------------------------------------
%Appendices
%----------------------------------------------------------------------------------------------
\NewDocumentEnvironment{singlePDFpage}{mmm}
{
\includepdf[pages=-,scale=.75,pagecommand={
\subsection{#1}\label{#2}
},linktodoc=true]
{appendix/#3}
}

\NewDocumentEnvironment{multiPDFpage}{mmm}
{
\includepdf[pages=1,scale=.75,pagecommand={
\subsection{#1}\label{#2}
},linktodoc=true]
{appendix/#3}}
{\includepdf[pages=2-,scale=0.85,pagecommand={},linktodoc=true]{appendix/#3}
}
