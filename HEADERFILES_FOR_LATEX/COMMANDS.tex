%This file contains all commands simplifying LaTeX text editing
%For each new command clearly document its dependencies (packages) and its use
%Adhere to the ordering by category!

%----------------------------------------------------------------------------------------------
%Maths
%----------------------------------------------------------------------------------------------
\renewcommand{\vec}[1]{\mathbf{#1}} %Notation of vectors
\newcommand{\vect}[1]{\mathbf{\dot{#1}}} %Notation of time derivative of a vector
\newcommand{\vectt}[1]{\mathbf{\ddot{#1}}} %Notation of the second time derivative of a vector
\newcommand{\vecd}[1]{\mathbf{#1'}} %Notation of the first derivative of a vector
\newcommand{\vecdd}[1]{\mathbf{#1''}} 
%In the following all vector definitions will be repeated with the inclusion of a prepended calligraphic letter to define the respective coordinate frame
\newcommand{\cvec}[2]{{}_\mathcal{#1}\mathbf{#2}} %Notation of vectors
\newcommand{\cvect}[2]{{}_\matcal{#1}\mathbf{\dot{#2}}} %Notation of time derivative of a vector
\newcommand{\cvectt}[2]{{}_\matcal{#1}\mathbf{\ddot{#2}}} %Notation of the second time derivative of a vector
\newcommand{\cvecd}[2]{{}_\matcal{#1}\mathbf{#2'}} %Notation of the first derivative of a vector
\newcommand{\cvecdd}[2]{{}_\matcal{#1}\mathbf{#2''}}

%Notation of the second derivative of a vector
\newcommand{\tens}[1]{\underline{\underline{#1}}} %Notation of a tensor
\newcommand{\vprod}[2]{\vec{#1}\times\vec{#2}} %Vector product
\newcommand{\inRone}[1]{\in\mathbb{R}^{#1}} %in 1D set of real numbers
\newcommand{\inRtwo}[2]{\in\mathbb{R}^{#1 \times #2}} %in 2D set of real numbers
\newcommand{\inRthree}[3]{\in\mathbb{R}^{#1 \times #2 \times #3}} %in 3D set of real numbers
\newcommand{\ddt}{\frac{d}{dt}} %time derivative of
\newcommand{\onha}{\frac{1}{2}} %one half
\newcommand{\onth}{\frac{1}{3}} %one third
\newcommand{\onfo}{\frac{1}{4}} %one forth
\newcommand{\prob}[1]{\mathbb{P}(#1)} %Probability of some event
\newcommand{\expe}[1]{\mathbb{E}[#1]} %Expected value for some stochastic variable
\newcommand{\var}[1]{\text{Var}(#1)} %Variance of some stochastic variable
\newcommand{\cov}[1]{\text{Cov}(#1)} %Covariance of some stochastic variables
\newcommand{\corr}[1]{\text{Corr}(#1)} %Correlation of some stochastic variables
\newcommand{\sign}[1]{\text{sign}(#1)} %signum function
\newcommand{\sinc}{\text{sinc}} %sinc function
\newcommand{\diag}{\text{diag}} %diagonal matrix

%----------------------------------------------------------------------------------------------
%Tables
%----------------------------------------------------------------------------------------------

\newcommand{\mc}[2]{\multicolumn{1}{#1}{#2}} %This command is a short for multicolumn
%Usage:
%Instead of \multicolumn{1}{<type>}{<content>}
%write \mc{<type>}{<content>}
%Usecase: When trying to remove vertical lines for only one line in your tabular environment
%multicolumn can come in handy.

%----------------------------------------------------------------------------------------------
% Formating
%----------------------------------------------------------------------------------------------

\newcommand{\sbs}[4]{\begin{minipage}[t!]{#1\linewidth}#3\end{minipage}\begin{minipage}[t!]{#2\linewidth}#4\end{minipage}} %Two minipages dividing the page in two scaled parts
%Usage:
%\sbs{<width of the left, for example 0.45>}{<width on the right, for example 0.45>}
%{
%<content on the left>
%}
%{
%<content on the right>
%}

\newcommand{\sbss}[2]{\sbs{0.45}{0.45}{#1}{#2}} %Two minipages dividing the page in two equal parts
%Usage:
%\sbss
%{
%<content on the left>
%}
%{
%<content on the right>
%}


\newcommand{\sbsss}[3]{\begin{minipage}[t!]{0.42\linewidth}#2\end{minipage}\hspace{#1 ex}\begin{minipage}[t!]{0.42\linewidth}#3\end{minipage}} %Two minipages dividing the page in two equal parts with some user-defined space in [ex] in between
%Usage:
%\sbsss{5}
%{
%<content on the left>
%}
%{
%<content on the right>
%}

%----------------------------------------------------------------------------------------------
%Colors
%----------------------------------------------------------------------------------------------
\definecolor{pantone2128}{RGB}{152,182,228}
\definecolor{pantone2727}{RGB}{48,127,226}
\definecolor{pantone2728}{RGB}{0,71,187}
\definecolor{darkblue}{RGB}{0,35,156}
\definecolor{pantone2757}{RGB}{0,30,96}

%----------------------------------------------------------------------------------------------
%Macros
%----------------------------------------------------------------------------------------------

\newcommand{\statenotation}{
\section{Notation}
\begin{center}
\begin{tabular}{p{0.3\linewidth}p{0.3\linewidth}l}
Operator&Meaning&Example\\
\midrule
\textbf{bold text}&Vector&$\vec{v}$\\
Prepended calligraphic letter&Vector or Matrix represented in the coordinate frame $\mathcal{A}$&$\cvec{A}{v}$\\
Dot above&Time derivative&$\dot{x}$\\
Prime&General derivative&$f'$\\
Double underline&Tensor&$\tens{I}$\\
\midrule
\end{tabular}
\end{center}
}
%Usage:
%Type \statenotation where you'd like to define your notation.

%Checkboxes
\usepackage{pifont}
\newcommand{\cmark}{\ding{51}}%
\newcommand{\xmark}{\ding{55}}%
\newcommand{\done}{\rlap{$\square$}{\raisebox{2pt}{\large\hspace{1pt}\cmark}}%
\hspace{-2.5pt}}
\newcommand{\wontfix}{\rlap{$\square$}{\large\hspace{1pt}\xmark}}