\documentclass[handout,usenames,dvipsnames]{beamer}
%use [handout] to get rid of uncovered slides
%Note that when updating the headerfiles xcolor and hyperref need to be removed again to avoid an option clash.
\usetheme{Boadilla}

\input{../HEADERFILES_FOR_LATEX/PACKAGES}
%This file contains all new environments simplifying LaTeX text editing
%For each new environment clearly document its dependencies (packages) and its use
%Adhere to the ordering by category!

%----------------------------------------------------------------------------------------------
%Title Page
%----------------------------------------------------------------------------------------------
\NewDocumentEnvironment{TTitlepage}{mmmm} %Defines an environment for the titlepage
{
\begin{titlepage}
\setlength{\parindent}{0pt}	 \centering
\Huge{\bfseries #1} \\						% X: Document Title
\huge Project TELL \\	[1.cm]

	\begin {table} [H]
		\label {tab:contenttable} \large 
		\centering
		\begin {tabular} {r l} \midrule

		\bfseries{Doc. Reference} & #2 \\ 		% X: Doc. Ref. No.
		\bfseries{Author} & #3  \\						% X: Author Name
		\bfseries{Date} & #4  \\ \\ 						% X: Date of creation

		\end {tabular}
	\end {table}

\begin{figure}[H]
	\centering
   	\includegraphics[width=7cm]{../HEADERFILES_FOR_LATEX/Logo/tell}
\end{figure}

	\vspace*{\fill}
	\begin {table} [H]
		\label {tab:notetable} \large \centering
		\begin {tabular} {l p{12cm}} 

		\midrule
		\bfseries{Note} 
							
		% X: Add notes here if you have any after the &, else write N/A
		&
}{
\end {tabular}
	\end {table}
\end{titlepage}
}
%Usage
%\begin{TTitlepage}{<Document Title>}{<Document Reference Number>}{<Author Name>}{<Date Of Creation>}
%<your notes on the document>
%\end{TTitlepage}
%Dependencies: \usepackage{booktabs,xparse}	
		

%----------------------------------------------------------------------------------------------
%Tables
%----------------------------------------------------------------------------------------------
\NewDocumentEnvironment{TTable}{O{1}mmmm} %Defines a basic tabular environment
{
\begin{table}[H]
\centering
\rowcolors{#1}{white}{gray!25}
\begin{tabular}{#2}
\toprule
}
{
\bottomrule
\end{tabular}
\caption[#3]{#4}
\label{#5}
\end{table}
}
%Usage:
%\begin{TTable}[<startindex for colors>]
%{<table layout>}
%{<your short caption>}
%{<your caption>}
%{<your label>}
%<your Content>
%\end{TTable}
%Dependencies: \usepackage{booktabs,xcolors,xparse}



\NewDocumentEnvironment{TTable*}{mmmm} %Defines a basic tabular environment
{
\begin{table}[H]
\centering
\begin{tabular}{#1}
\toprule
}
{
\bottomrule
\end{tabular}
\caption[#2]{#3}
\label{#4}
\end{table}
}
%Usage:
%\begin{TTable*}{<table layout>}
%{<your short caption>}
%{<your caption>}
%{<your label>}

%<your Content>

%\end{TTable*}



\NewDocumentEnvironment{TSimTable}{mmm} %Defines a tabular environment for simulation data
{
\tiny
\begin{landscape} %rotate the table by 90 degrees
\begin{TTable}[8]{l|l|l|l|l|l|l|l|l|l|l|l|l|l}{#1}{#2}{#3} %Use the TTable environment

%First line of the table head
\multicolumn{2}{l}{}&\multicolumn{7}{l}{\textbf{\large Parameters}}&\multicolumn{5}{l}{\textbf{\large Results}}\\
\cmidrule[2pt]{3-6}\cmidrule[2pt]{10-12}
%Second line of the table head
\multicolumn{2}{l}{}&\multicolumn{3}{l}{Rocket}&\multicolumn{3}{l}{Motor}&\multicolumn{1}{l}{}&\multicolumn{2}{l}{Performance}&\multicolumn{3}{l}{Stability}\\
\cmidrule{3-14}
%Third line of the table head
\mc{l}{SIN}&\mc{l}{Software}&\mc{l}{Name}&\mc{l}{Conf.}&\mc{l}{$m_{tot}$}&\mc{l}{Type}&\mc{l}{$m_{Motor}$}&\mc{l}{$m_{Fuel}$}&\mc{l}{$m_{Stage}$}&\mc{l}{$h$}&\mc{l}{$a_{max}$}&\mc{l}{$v_{LRE}$}&\mc{l}{St. Mar.}&\mc{l}{Dpg. Ratio}\\ %the \mc command gets rid of unneeded vertical lines, see COMMANDS for an explanation
%Forth line of the table head
\multicolumn{4}{l}{}&\mc{l}{$[\si{\kilo\gram}]$}&\mc{l}{}&\mc{l}{$[\si{\kilo\gram}]$}&\mc{l}{$[\si{\kilo\gram}]$}&\mc{l}{$[\si{\kilo\gram}]$}&\mc{l}{$[\si{\kilo\meter}]$}&\mc{l}{$[\si{\meter\per\second\squared}]$}&\mc{l}{$[\si{\meter\per\second}]$}&\multicolumn{2}{l}{}\\
\midrule
}
{
\end{TTable}
\end{landscape}
\normalsize
}
%Usage
%\begin{TSimTable}{<your short caption>}{<your caption>}{<your label>}
%<your content in 14 columns>
%\end{TSimTable}
%Dependencies: \usepackage{lscape,booktabs,siunitx,xcolor,xparse}

\NewDocumentEnvironment{TDefinitionTable}{} %Used to define variables in an equation
{
\footnotesize
\begin{center}
\begin{tabular}{l@{$\quad$}l@{\dotfill$\quad$}cl}
}
{
\end{tabular}
\end{center}
\normalsize
}
%Usage:
%\begin{TDefinitionTable}
%$\vect{P}$&change of linear momentum &in& $\si{\kilo\gram\meter\per\second\squared}$\\
%$m$&mass&in&$\si{\kilo\gram}$\\
%$\vec{a}$&acceleration&in&$\si{\meter\per\second\squared}$\\
%$\vec{F}$&resultant force&in&$\si{N}=\si{\kilo\gram\meter\per\second\squared}$
%\end{TDefinitionTable}

\NewDocumentEnvironment{TValueTable}{} %Used to introduce a series of parameters with certain values
{
\footnotesize
\begin{center}
\begin{tabular}{r@{ = }lr}
}
{
\end{tabular}
\end{center}
\normalsize
}
%Usage:
%\begin{TValueTable}
%$h_E$&$\SI{32}{\milli\meter}$&Extension of the brakes\\
%$h_p$&$\SI{32}{\milli\meter}$&Height of a single plate\\
%$l_B$&$\SI{80}{\milli\meter}$&Width of one brake\\
%$n_p$&$l_B/l_p=\SI{5}{}$&Number of plates\\
%$A_p$&$h_p\cdot l_p\cdot n_p=\SI{25.6}{\centi\meter\squared}$&Total area of the plates\\
%$A_c$&$\SI{26.8}{\centi\meter\squared}$&Actual area of the brake
%\end{TValueTable}


%----------------------------------------------------------------------------------------------
%Appendices
%----------------------------------------------------------------------------------------------
\NewDocumentEnvironment{singlePDFpage}{mmm}
{
\includepdf[pages=-,scale=.75,pagecommand={
\subsection{#1}\label{#2}
},linktodoc=true]
{appendix/#3}
}

\NewDocumentEnvironment{multiPDFpage}{mmm}
{
\includepdf[pages=1,scale=.75,pagecommand={
\subsection{#1}\label{#2}
},linktodoc=true]
{appendix/#3}}
{\includepdf[pages=2-,scale=0.85,pagecommand={},linktodoc=true]{appendix/#3}
}

%This document contains all specially defined environments, used for the presentation template
%The prepended T stands for "Template"
%The prepended TP stands for "TemplatePresentation" - it is the general identifier for environments in this template
%The prepended TF stands for "TemplateFrame" and denotes all environments that create a frame of their own. Multiple commands exist in parallel, as TP and TF, such that they can be combined flexibly.

\usepackage{environ} %Define new environments that work with frames
%Usage for frames:
%\NewEnviron{EnvironmentName}[<#arguments>][<default of optional arguments starting with #1>]{%
%\begin{frame}{<FrameTitle>}
%<macrobeforebody>
%\BODY
%<macroafterbody>
%\end{frame}
%}


\NewEnviron{TFTimeSchedule}[1][]{
\begin{frame}{Time Schedule#1}
\begin{itemize}
\BODY
\end{itemize}
\end{frame}
}
%Usage:
%\begin{TPTimeSchedule}
%\item 15' This
%\item 70' That
%\item 5'  And others
%\end{TPTimeSchedule}

\NewEnviron{TFLearningObjectives}[1][]{
\begin{frame}{Learning Objectives#1}
\begin{itemize}
\BODY
\end{itemize}
\end{frame}
}
%Usage:
%\begin{TPLearningObjectives}
%\item Understand the assumptions for the Euler equations.
%\item Be able to apply the Euler equations to 3D rigid bodies.
%\item Understand the TSP-Rule.
%\end{TPLearningObjectives}

\NewEnviron{TFPicture}[3][]{
\begin{frame}
\begin{figure}
\includegraphics[width=\linewidth,height=0.8\textheight,keepaspectratio]{#1}
\caption{#2}
\label{#3}
\end{figure}
\BODY
\end{frame}
}
%Usage:
%\begin{TFPicture}[Landscape]{An image in landscape orientation.}{fig:Landscape}
%Some text below.
%\end{TFPicture}

\NewEnviron{TPPicture}[3][]{
\begin{figure}
\includegraphics[width=\linewidth,height=0.65\textheight,keepaspectratio]{#1}
\caption{#2}
\label{#3}
\end{figure}
\BODY
}
%Usage:
%\begin{TPPicture}[Landscape]{An image in landscape orientation.}{fig:Landscape}
%Some text below.
%\end{TPPicture}

\NewEnviron{TFTwoColumns}[3][]{
\begin{frame}{#1}
\begin{columns}[T]
\begin{column}{.48\textwidth}
#2
\end{column}%
\hfill%
\begin{column}{.48\textwidth}
#3
\end{column}
\end{columns}
\end{frame}
}
%Usage:
%\begin{TPTwoColumns}[Title]
%{content on the left}
%{and also content on the right}
%\end{TPTwoColumns}

\newcommand{\matlab}{
\lstset{language=Matlab,%
%basicstyle=\color{red},
breaklines=true,%
morekeywords={matlab2tikz},
keywordstyle=\color{blue},%
morekeywords=[2]{1}, keywordstyle=[2]{\color{black}},
identifierstyle=\color{black},%
stringstyle=\color{BlueViolet},
commentstyle=\color{ForestGreen},%
showstringspaces=false,%without this there will be a symbol in the places where there is a space
numbers=left,%
numberstyle={\tiny \color{black}},% size of the numbers
numbersep=9pt, % this defines how far the numbers are from the text
backgroundcolor=\color{yellow!20!white},
frame=single,
escapeinside={\%*}{*)},
}
}
%This file contains all commands simplifying LaTeX text editing
%For each new command clearly document its dependencies (packages) and its use
%Adhere to the ordering by category!

%----------------------------------------------------------------------------------------------
%Maths
%----------------------------------------------------------------------------------------------
\renewcommand{\vec}[1]{\mathbf{#1}} %Notation of vectors
\newcommand{\vect}[1]{\mathbf{\dot{#1}}} %Notation of time derivative of a vector
\newcommand{\vectt}[1]{\mathbf{\ddot{#1}}} %Notation of the second time derivative of a vector
\newcommand{\vecd}[1]{\mathbf{#1'}} %Notation of the first derivative of a vector
\newcommand{\vecdd}[1]{\mathbf{#1''}} 
%In the following all vector definitions will be repeated with the inclusion of a prepended calligraphic letter to define the respective coordinate frame
\newcommand{\cvec}[2]{{}_\mathcal{#1}\mathbf{#2}} %Notation of vectors
\newcommand{\cvect}[2]{{}_\matcal{#1}\mathbf{\dot{#2}}} %Notation of time derivative of a vector
\newcommand{\cvectt}[2]{{}_\matcal{#1}\mathbf{\ddot{#2}}} %Notation of the second time derivative of a vector
\newcommand{\cvecd}[2]{{}_\matcal{#1}\mathbf{#2'}} %Notation of the first derivative of a vector
\newcommand{\cvecdd}[2]{{}_\matcal{#1}\mathbf{#2''}}

%Notation of the second derivative of a vector
\newcommand{\tens}[1]{\underline{\underline{#1}}} %Notation of a tensor
\newcommand{\vprod}[2]{\vec{#1}\times\vec{#2}} %Vector product
\newcommand{\inRone}[1]{\in\mathbb{R}^{#1}} %in 1D set of real numbers
\newcommand{\inRtwo}[2]{\in\mathbb{R}^{#1 \times #2}} %in 2D set of real numbers
\newcommand{\inRthree}[3]{\in\mathbb{R}^{#1 \times #2 \times #3}} %in 3D set of real numbers
\newcommand{\ddt}{\frac{d}{dt}} %time derivative of
\newcommand{\onha}{\frac{1}{2}} %one half
\newcommand{\onth}{\frac{1}{3}} %one third
\newcommand{\onfo}{\frac{1}{4}} %one forth
\newcommand{\prob}[1]{\mathbb{P}(#1)} %Probability of some event
\newcommand{\expe}[1]{\mathbb{E}[#1]} %Expected value for some stochastic variable
\newcommand{\var}[1]{\text{Var}(#1)} %Variance of some stochastic variable
\newcommand{\cov}[1]{\text{Cov}(#1)} %Covariance of some stochastic variables
\newcommand{\corr}[1]{\text{Corr}(#1)} %Correlation of some stochastic variables
\newcommand{\sign}[1]{\text{sign}(#1)} %signum function
\newcommand{\sinc}{\text{sinc}} %sinc function
\newcommand{\diag}{\text{diag}} %diagonal matrix

%----------------------------------------------------------------------------------------------
%Tables
%----------------------------------------------------------------------------------------------

\newcommand{\mc}[2]{\multicolumn{1}{#1}{#2}} %This command is a short for multicolumn
%Usage:
%Instead of \multicolumn{1}{<type>}{<content>}
%write \mc{<type>}{<content>}
%Usecase: When trying to remove vertical lines for only one line in your tabular environment
%multicolumn can come in handy.

%----------------------------------------------------------------------------------------------
% Formating
%----------------------------------------------------------------------------------------------

\newcommand{\sbs}[4]{\begin{minipage}[t!]{#1\linewidth}#3\end{minipage}\begin{minipage}[t!]{#2\linewidth}#4\end{minipage}} %Two minipages dividing the page in two scaled parts
%Usage:
%\sbs{<width of the left, for example 0.45>}{<width on the right, for example 0.45>}
%{
%<content on the left>
%}
%{
%<content on the right>
%}

\newcommand{\sbss}[2]{\sbs{0.45}{0.45}{#1}{#2}} %Two minipages dividing the page in two equal parts
%Usage:
%\sbss
%{
%<content on the left>
%}
%{
%<content on the right>
%}


\newcommand{\sbsss}[3]{\begin{minipage}[t!]{0.42\linewidth}#2\end{minipage}\hspace{#1 ex}\begin{minipage}[t!]{0.42\linewidth}#3\end{minipage}} %Two minipages dividing the page in two equal parts with some user-defined space in [ex] in between
%Usage:
%\sbsss{5}
%{
%<content on the left>
%}
%{
%<content on the right>
%}

%----------------------------------------------------------------------------------------------
%Colors
%----------------------------------------------------------------------------------------------
\definecolor{pantone2128}{RGB}{152,182,228}
\definecolor{pantone2727}{RGB}{48,127,226}
\definecolor{pantone2728}{RGB}{0,71,187}
\definecolor{darkblue}{RGB}{0,35,156}
\definecolor{pantone2757}{RGB}{0,30,96}

%----------------------------------------------------------------------------------------------
%Macros
%----------------------------------------------------------------------------------------------

\newcommand{\statenotation}{
\section{Notation}
\begin{center}
\begin{tabular}{p{0.3\linewidth}p{0.3\linewidth}l}
Operator&Meaning&Example\\
\midrule
\textbf{bold text}&Vector&$\vec{v}$\\
Prepended calligraphic letter&Vector or Matrix represented in the coordinate frame $\mathcal{A}$&$\cvec{A}{v}$\\
Dot above&Time derivative&$\dot{x}$\\
Prime&General derivative&$f'$\\
Double underline&Tensor&$\tens{I}$\\
\midrule
\end{tabular}
\end{center}
}
%Usage:
%Type \statenotation where you'd like to define your notation.

%Checkboxes
\usepackage{pifont}
\newcommand{\cmark}{\ding{51}}%
\newcommand{\xmark}{\ding{55}}%
\newcommand{\done}{\rlap{$\square$}{\raisebox{2pt}{\large\hspace{1pt}\cmark}}%
\hspace{-2.5pt}}
\newcommand{\wontfix}{\rlap{$\square$}{\large\hspace{1pt}\xmark}}
\graphicspath{{./Pictures/}}

\title{Exercise Week 01}
\author{GianAndrea Müller\\ \url{mailto:muellegi@student.ethz}}
\date{\today}

\begin{document}
%Make title page from above supplied information on title, author, etc.
\maketitle

\begin{TFTimeSchedule}
\item 10' Generelle Information
\item 15' Ganzzahldivision und Modulo mit Übung
\item 10' Repetition einfacher Ausdrücke
\item 15' Binäre Darstellung von Zahlen 
\item 15' Pause
\item 40' Übungsbearbeitung
\end{TFTimeSchedule}

\begin{TFLearningObjectives}
\item Organisation: Klärung aller Fragen bezüglich der Übungsumgebung.
\item Verständnis: Ganzzahldivision
\item Verständnis: Ausdrücke
\item Verständnis: Binäre Darstellung von Zahlen
\end{TFLearningObjectives}

\begin{frame}{Allgemeines}
\begin{itemize}
\item Selbstverantwortung übernehmen.
\item Programmieren ist eine Chance.
\item Let me \easyurl{google}{www.google.com} that for you.
\end{itemize}
\end{frame}

\begin{frame}{Organisation}

\begin{block}{Wieso Übungen lösen?}
\begin{itemize}
\item Erfahrung sammeln.
\item XP sammeln.
\item Bonusaufgaben lösen.
\item 0.25 Notenpunkte gewinnen.
\end{itemize}
\end{block}

\vfill

\easyurl{Polybox}{https://polybox.ethz.ch/index.php/s/1bgdJMjROEzGNN6}

\easyurl{First login}{https://expert.ethz.ch/mavt18}

\easyurl{All exercises}{https://expert.ethz.ch/}

\easyurl{C++ reference}{http://www.cplusplus.com}

\easyurl{Hardcore C++ reference}{http://de.cppreference.com/w/}
\end{frame}

\begin{frame}[fragile]{Ein C++ Programm}
\begin{TFCpp}
#include <iostream>

using namespace std;

int main (){

	int a; 
	int b = 4;
	cin>>a;
	cout<<a+b<<endl;

	return 0;
}
\end{TFCpp}
\end{frame}

\begin{frame}[fragile]{Ein C++ Programm}
\begin{TFCpp}
#include <iostream> //Bibliothek

using namespace std; //Loesung fuer std::cin, std::cin

int main (){ //Beginn der Hauptfunktion

	int a; //Variabeldefinition ohne Initialisierung
	int b = 4; //Variabeldefinition mit Initialisierung
	cin>>a; //Einlesen mit cin>>
	cout<<a+b<<endl; //Ausgeben mit cout<<
	//Neue Zeile mit endl
	return 0;//Rueckgabewert der Hauptfunktion
}
\end{TFCpp}
\end{frame}

\begin{frame}[fragile]{Ganzzahldivision}

\begin{TFCpp}
using namespace std;

int main (){
	int a = 6;
	int b = 4;
	cout<<a/b<<endl;
	return 0;
}
\end{TFCpp}

\uncover<2>{
\begin{block}{Keep in mind!}
Eine Ganzzahldivision löscht die Nachkommastellen.
\end{block}
}
\end{frame}

\begin{frame}[fragile]{Integer division, examples}

\begin{TFCpp}
using namespace std;

int main (){
	int a = 6;
	int b = 4;
	cout<<b/a<<endl;
	cout<<1/2<<endl;
	cout<<double(1)/2<<endl;
	cout<<1.0/2<<endl;
	return 0;
}
\end{TFCpp}
\end{frame}

\begin{frame}[fragile]{Ganzzahldivision, Beispiele}

\begin{TFCpp}
using namespace std;

int main (){
	int a = 6;
	int b = 4;
	cout<<b/a<<endl;	//0
	cout<<1/2<<endl;	//0
	cout<<double(1)/2<<endl;	//0.5
	cout<<1.0/2<<endl;	//0.5
	return 0;
}
\end{TFCpp}
\end{frame}


\begin{frame}[fragile]{Modulo}
\begin{TFCpp}
using namespace std;

int main (){
	int a = 6;
	int b = 4;
	cout<<a%b<<endl;
	return 0;
}
\end{TFCpp}

\uncover<2>{
\begin{block}{Keep in mind!}
The modulo operation returns the rest of an integer division.
\end{block}
\href{https://codeboard.io/projects/75269}{\textcolor{blue}{\underline{Short demonstration}}}
}
\end{frame}

\begin{frame}{Exercise 01\_1 $\sim$ 5'}
Write a program which reads in two integers $a$ and $b$, then calculates the quotient $\frac{a}{b}$ as a mixed expression and outputs it. 

\vspace{1.5ex}

For example, if $a = 17$ and $b = 6$, the output should be $2\ 5/6$.
\end{frame}

\begin{frame}{Solution 01\_1}
\href{https://codeboard.io/projects/75270}{\textcolor{blue}{\underline{Solution to 01\_1}}}
\end{frame}

\begin{frame}{L- und R-Werte}
\begin{block}{Definition}
Ein L-Wert ist ein Ausdruck der über eine Adresse im Computerspeicher verfügt und somit auf der linken Seite eines Zuweisungsoperators (=) stehen kann. 

Alle anderen Ausdrücke sind R-Werte.
\end{block}
\end{frame}

\begin{frame}[fragile]{L- und R-Werte}
\begin{TFCpp}
using namespace std;

int main (){
	int a = 6;
	int b = 4;
	//5 = a;
	//(1+a) = 7;
	int c = a + b;
	cout<<a%b<<endl;
	return 0;
}
\end{TFCpp}
\end{frame}

\begin{frame}[fragile]{Exercise 01\_2 ~5'}
\begin{block}{Evaluating expressions}
\begin{enumerate}
\item Which of the following character sequences are not C++ expressions and why not?
\begin{enumerate}
\item \verb+1*(2*3)+
\item \verb+(a=1)+
\item \verb+(1+
\item \verb+(a*3)=(b*5)+
\end{enumerate}
\item For all of the expressions that you identified in 1), decide whether these are lvalues or rvalues.
\item Determine the values of the expressions and explain how these values are obtained.
\end{enumerate}
\end{block}

\vfill

\easyurl{Expressions}{http://en.cppreference.com/w/cpp/language/expressions}

\easyurl{Precedence}{http://en.cppreference.com/w/cpp/language/operator_precedence}
\end{frame}

\begin{frame}[fragile]{Solution 01\_2}
\easyurl{Solution 01\_2}{https://codeboard.io/projects/75273}
\end{frame}

\begin{TFTwoColumns}[Binäre Darstellung]{
\begin{align*}
91310&=\uncover<2-5>{10*9131+&0}\\
\uncover<3-5>{9131&=10*913+&1}\\
\uncover<4-5>{913&=10*91+&3\\
91&=10*9+&1\\
9&=10*0+&9}
\end{align*}
}
{
\begin{align*}
\uncover<5>{61&=2*30+&1\\
30&=2*15+&0\\
15&=2*7+&1\\
7&=2*3+&1\\
3&=2*1+&1\\
1&=2*0+&1}
\end{align*}
}
\end{TFTwoColumns}

\begin{frame}{Binäre Darstellung}
\begin{tabular}{lllllll}
&&&&&&$\sum$\\
Ziffern&9&1&3&1&0\\
Multiplikator&\uncover<2-4>{10000&1000&100&10&1}\\
Wert&\uncover<3-4>{90000&1000&300&10&0&91310}
\end{tabular}

\vfill

\begin{tabular}{llllllll}
\uncover<4>{&&&&&&&$\sum$\\
Ziffern&1&1&1&1&0&1\\
Multiplikator&32&16&8&4&2&1\\
Wert&32&16&8&4&0&1&61\\}
\end{tabular}
\end{frame}

\begin{frame}{Negative Binärzahlen}
\begin{center}
\begin{tabular}{rrrrrr}
bin&uint&int&bin&uint&int\\\midrule
0000&0&0&1000&8&-8\\
0001&1&1&1001&9&-7\\
0010&2&2&1010&10&-6\\
0011&3&3&1011&11&-5\\
0100&4&4&1100&12&-4\\
0101&5&5&1101&13&-3\\
0110&6&6&1110&14&-2\\
0111&7&7&1111&15&-1
\end{tabular}
\end{center}

\vfill

\easyurl{One's complement}{https://en.wikipedia.org/wiki/Ones\%27_complement} \hfill \easyurl{Demo}{https://codeboard.io/projects/75301} \hfill \easyurl{Two's complement}{https://en.wikipedia.org/wiki/Two\%27s_complement}
\end{frame}

\begin{frame}{Tips für Exercise 3}
\begin{itemize}
\item Serienschaltung von $R_1$ und $R_2$: $R_{tot}=R_1+R_2$
\item Parallelschaltung von $R_1$ und $R_2$: $R_{tot}=\frac{R_1\cdot R_2}{R_1+R_2}$
\item Ganzzahlrundung: $1.999 \rightarrow 1$
\item Arithmetische Rundung: $1.5\rightarrow 2$
\end{itemize}
\end{frame}
\end{document}