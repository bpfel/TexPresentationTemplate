%===========================================================================
%generation of assignment and solution in one .tex file. make sure that write18 is activated(--enable-write18 for example as you call pdflatex.exe)

\ifx\conditionmacro\undefined 
\immediate\write18{%
	pdfLaTeX --jobname="presentation_handout"
	\gdef\string\conditionmacro{1}\string\input\space\jobname
}%
\immediate\write18{%
	pdfLaTeX --jobname="presentation_with_notes"
	\gdef\string\conditionmacro{3}\string\input\space\jobname
}%
\gdef\conditionmacro{2}
%\expandafter\stop
\fi
%---------------------------------------------------------------------------

\ifnum\conditionmacro=1 \documentclass[handout,usenames,dvipsnames]{beamer}\fi
\ifnum\conditionmacro=2 \documentclass[usenames,dvipsnames]{beamer} \fi
\ifnum\conditionmacro=3 \documentclass[notes,handout,usenames,dvipsnames]{beamer}
\usepackage{pgfpages}
\setbeameroption{show notes on second screen=left} \fi

\usetheme{Boadilla}

\input{../HEADERFILES_FOR_LATEX/PACKAGES}
%This file contains all new environments simplifying LaTeX text editing
%For each new environment clearly document its dependencies (packages) and its use
%Adhere to the ordering by category!

%----------------------------------------------------------------------------------------------
%Title Page
%----------------------------------------------------------------------------------------------
\NewDocumentEnvironment{TTitlepage}{mmmm} %Defines an environment for the titlepage
{
\begin{titlepage}
\setlength{\parindent}{0pt}	 \centering
\Huge{\bfseries #1} \\						% X: Document Title
\huge Project TELL \\	[1.cm]

	\begin {table} [H]
		\label {tab:contenttable} \large 
		\centering
		\begin {tabular} {r l} \midrule

		\bfseries{Doc. Reference} & #2 \\ 		% X: Doc. Ref. No.
		\bfseries{Author} & #3  \\						% X: Author Name
		\bfseries{Date} & #4  \\ \\ 						% X: Date of creation

		\end {tabular}
	\end {table}

\begin{figure}[H]
	\centering
   	\includegraphics[width=7cm]{../HEADERFILES_FOR_LATEX/Logo/tell}
\end{figure}

	\vspace*{\fill}
	\begin {table} [H]
		\label {tab:notetable} \large \centering
		\begin {tabular} {l p{12cm}} 

		\midrule
		\bfseries{Note} 
							
		% X: Add notes here if you have any after the &, else write N/A
		&
}{
\end {tabular}
	\end {table}
\end{titlepage}
}
%Usage
%\begin{TTitlepage}{<Document Title>}{<Document Reference Number>}{<Author Name>}{<Date Of Creation>}
%<your notes on the document>
%\end{TTitlepage}
%Dependencies: \usepackage{booktabs,xparse}	
		

%----------------------------------------------------------------------------------------------
%Tables
%----------------------------------------------------------------------------------------------
\NewDocumentEnvironment{TTable}{O{1}mmmm} %Defines a basic tabular environment
{
\begin{table}[H]
\centering
\rowcolors{#1}{white}{gray!25}
\begin{tabular}{#2}
\toprule
}
{
\bottomrule
\end{tabular}
\caption[#3]{#4}
\label{#5}
\end{table}
}
%Usage:
%\begin{TTable}[<startindex for colors>]
%{<table layout>}
%{<your short caption>}
%{<your caption>}
%{<your label>}
%<your Content>
%\end{TTable}
%Dependencies: \usepackage{booktabs,xcolors,xparse}



\NewDocumentEnvironment{TTable*}{mmmm} %Defines a basic tabular environment
{
\begin{table}[H]
\centering
\begin{tabular}{#1}
\toprule
}
{
\bottomrule
\end{tabular}
\caption[#2]{#3}
\label{#4}
\end{table}
}
%Usage:
%\begin{TTable*}{<table layout>}
%{<your short caption>}
%{<your caption>}
%{<your label>}

%<your Content>

%\end{TTable*}



\NewDocumentEnvironment{TSimTable}{mmm} %Defines a tabular environment for simulation data
{
\tiny
\begin{landscape} %rotate the table by 90 degrees
\begin{TTable}[8]{l|l|l|l|l|l|l|l|l|l|l|l|l|l}{#1}{#2}{#3} %Use the TTable environment

%First line of the table head
\multicolumn{2}{l}{}&\multicolumn{7}{l}{\textbf{\large Parameters}}&\multicolumn{5}{l}{\textbf{\large Results}}\\
\cmidrule[2pt]{3-6}\cmidrule[2pt]{10-12}
%Second line of the table head
\multicolumn{2}{l}{}&\multicolumn{3}{l}{Rocket}&\multicolumn{3}{l}{Motor}&\multicolumn{1}{l}{}&\multicolumn{2}{l}{Performance}&\multicolumn{3}{l}{Stability}\\
\cmidrule{3-14}
%Third line of the table head
\mc{l}{SIN}&\mc{l}{Software}&\mc{l}{Name}&\mc{l}{Conf.}&\mc{l}{$m_{tot}$}&\mc{l}{Type}&\mc{l}{$m_{Motor}$}&\mc{l}{$m_{Fuel}$}&\mc{l}{$m_{Stage}$}&\mc{l}{$h$}&\mc{l}{$a_{max}$}&\mc{l}{$v_{LRE}$}&\mc{l}{St. Mar.}&\mc{l}{Dpg. Ratio}\\ %the \mc command gets rid of unneeded vertical lines, see COMMANDS for an explanation
%Forth line of the table head
\multicolumn{4}{l}{}&\mc{l}{$[\si{\kilo\gram}]$}&\mc{l}{}&\mc{l}{$[\si{\kilo\gram}]$}&\mc{l}{$[\si{\kilo\gram}]$}&\mc{l}{$[\si{\kilo\gram}]$}&\mc{l}{$[\si{\kilo\meter}]$}&\mc{l}{$[\si{\meter\per\second\squared}]$}&\mc{l}{$[\si{\meter\per\second}]$}&\multicolumn{2}{l}{}\\
\midrule
}
{
\end{TTable}
\end{landscape}
\normalsize
}
%Usage
%\begin{TSimTable}{<your short caption>}{<your caption>}{<your label>}
%<your content in 14 columns>
%\end{TSimTable}
%Dependencies: \usepackage{lscape,booktabs,siunitx,xcolor,xparse}

\NewDocumentEnvironment{TDefinitionTable}{} %Used to define variables in an equation
{
\footnotesize
\begin{center}
\begin{tabular}{l@{$\quad$}l@{\dotfill$\quad$}cl}
}
{
\end{tabular}
\end{center}
\normalsize
}
%Usage:
%\begin{TDefinitionTable}
%$\vect{P}$&change of linear momentum &in& $\si{\kilo\gram\meter\per\second\squared}$\\
%$m$&mass&in&$\si{\kilo\gram}$\\
%$\vec{a}$&acceleration&in&$\si{\meter\per\second\squared}$\\
%$\vec{F}$&resultant force&in&$\si{N}=\si{\kilo\gram\meter\per\second\squared}$
%\end{TDefinitionTable}

\NewDocumentEnvironment{TValueTable}{} %Used to introduce a series of parameters with certain values
{
\footnotesize
\begin{center}
\begin{tabular}{r@{ = }lr}
}
{
\end{tabular}
\end{center}
\normalsize
}
%Usage:
%\begin{TValueTable}
%$h_E$&$\SI{32}{\milli\meter}$&Extension of the brakes\\
%$h_p$&$\SI{32}{\milli\meter}$&Height of a single plate\\
%$l_B$&$\SI{80}{\milli\meter}$&Width of one brake\\
%$n_p$&$l_B/l_p=\SI{5}{}$&Number of plates\\
%$A_p$&$h_p\cdot l_p\cdot n_p=\SI{25.6}{\centi\meter\squared}$&Total area of the plates\\
%$A_c$&$\SI{26.8}{\centi\meter\squared}$&Actual area of the brake
%\end{TValueTable}


%----------------------------------------------------------------------------------------------
%Appendices
%----------------------------------------------------------------------------------------------
\NewDocumentEnvironment{singlePDFpage}{mmm}
{
\includepdf[pages=-,scale=.75,pagecommand={
\subsection{#1}\label{#2}
},linktodoc=true]
{appendix/#3}
}

\NewDocumentEnvironment{multiPDFpage}{mmm}
{
\includepdf[pages=1,scale=.75,pagecommand={
\subsection{#1}\label{#2}
},linktodoc=true]
{appendix/#3}}
{\includepdf[pages=2-,scale=0.85,pagecommand={},linktodoc=true]{appendix/#3}
}

%This document contains all specially defined environments, used for the presentation template
%The prepended T stands for "Template"
%The prepended TP stands for "TemplatePresentation" - it is the general identifier for environments in this template
%The prepended TF stands for "TemplateFrame" and denotes all environments that create a frame of their own. Multiple commands exist in parallel, as TP and TF, such that they can be combined flexibly.

\usepackage{environ} %Define new environments that work with frames
%Usage for frames:
%\NewEnviron{EnvironmentName}[<#arguments>][<default of optional arguments starting with #1>]{%
%\begin{frame}{<FrameTitle>}
%<macrobeforebody>
%\BODY
%<macroafterbody>
%\end{frame}
%}


\NewEnviron{TFTimeSchedule}[1][]{
\begin{frame}{Time Schedule#1}
\begin{itemize}
\BODY
\end{itemize}
\end{frame}
}
%Usage:
%\begin{TPTimeSchedule}
%\item 15' This
%\item 70' That
%\item 5'  And others
%\end{TPTimeSchedule}

\NewEnviron{TFLearningObjectives}[1][]{
\begin{frame}{Learning Objectives#1}
\begin{itemize}
\BODY
\end{itemize}
\end{frame}
}
%Usage:
%\begin{TPLearningObjectives}
%\item Understand the assumptions for the Euler equations.
%\item Be able to apply the Euler equations to 3D rigid bodies.
%\item Understand the TSP-Rule.
%\end{TPLearningObjectives}

\NewEnviron{TFPicture}[3][]{
\begin{frame}
\begin{figure}
\includegraphics[width=\linewidth,height=0.8\textheight,keepaspectratio]{#1}
\caption{#2}
\label{#3}
\end{figure}
\BODY
\end{frame}
}
%Usage:
%\begin{TFPicture}[Landscape]{An image in landscape orientation.}{fig:Landscape}
%Some text below.
%\end{TFPicture}

\NewEnviron{TPPicture}[3][]{
\begin{figure}
\includegraphics[width=\linewidth,height=0.65\textheight,keepaspectratio]{#1}
\caption{#2}
\label{#3}
\end{figure}
\BODY
}
%Usage:
%\begin{TPPicture}[Landscape]{An image in landscape orientation.}{fig:Landscape}
%Some text below.
%\end{TPPicture}

\NewEnviron{TFTwoColumns}[3][]{
\begin{frame}{#1}
\begin{columns}[T]
\begin{column}{.48\textwidth}
#2
\end{column}%
\hfill%
\begin{column}{.48\textwidth}
#3
\end{column}
\end{columns}
\end{frame}
}
%Usage:
%\begin{TPTwoColumns}[Title]
%{content on the left}
%{and also content on the right}
%\end{TPTwoColumns}

\newcommand{\matlab}{
\lstset{language=Matlab,%
%basicstyle=\color{red},
breaklines=true,%
morekeywords={matlab2tikz},
keywordstyle=\color{blue},%
morekeywords=[2]{1}, keywordstyle=[2]{\color{black}},
identifierstyle=\color{black},%
stringstyle=\color{BlueViolet},
commentstyle=\color{ForestGreen},%
showstringspaces=false,%without this there will be a symbol in the places where there is a space
numbers=left,%
numberstyle={\tiny \color{black}},% size of the numbers
numbersep=9pt, % this defines how far the numbers are from the text
backgroundcolor=\color{yellow!20!white},
frame=single,
escapeinside={\%*}{*)},
}
}
%This file contains all commands simplifying LaTeX text editing
%For each new command clearly document its dependencies (packages) and its use
%Adhere to the ordering by category!

%----------------------------------------------------------------------------------------------
%Maths
%----------------------------------------------------------------------------------------------
\renewcommand{\vec}[1]{\mathbf{#1}} %Notation of vectors
\newcommand{\vect}[1]{\mathbf{\dot{#1}}} %Notation of time derivative of a vector
\newcommand{\vectt}[1]{\mathbf{\ddot{#1}}} %Notation of the second time derivative of a vector
\newcommand{\vecd}[1]{\mathbf{#1'}} %Notation of the first derivative of a vector
\newcommand{\vecdd}[1]{\mathbf{#1''}} 
%In the following all vector definitions will be repeated with the inclusion of a prepended calligraphic letter to define the respective coordinate frame
\newcommand{\cvec}[2]{{}_\mathcal{#1}\mathbf{#2}} %Notation of vectors
\newcommand{\cvect}[2]{{}_\matcal{#1}\mathbf{\dot{#2}}} %Notation of time derivative of a vector
\newcommand{\cvectt}[2]{{}_\matcal{#1}\mathbf{\ddot{#2}}} %Notation of the second time derivative of a vector
\newcommand{\cvecd}[2]{{}_\matcal{#1}\mathbf{#2'}} %Notation of the first derivative of a vector
\newcommand{\cvecdd}[2]{{}_\matcal{#1}\mathbf{#2''}}

%Notation of the second derivative of a vector
\newcommand{\tens}[1]{\underline{\underline{#1}}} %Notation of a tensor
\newcommand{\vprod}[2]{\vec{#1}\times\vec{#2}} %Vector product
\newcommand{\inRone}[1]{\in\mathbb{R}^{#1}} %in 1D set of real numbers
\newcommand{\inRtwo}[2]{\in\mathbb{R}^{#1 \times #2}} %in 2D set of real numbers
\newcommand{\inRthree}[3]{\in\mathbb{R}^{#1 \times #2 \times #3}} %in 3D set of real numbers
\newcommand{\ddt}{\frac{d}{dt}} %time derivative of
\newcommand{\onha}{\frac{1}{2}} %one half
\newcommand{\onth}{\frac{1}{3}} %one third
\newcommand{\onfo}{\frac{1}{4}} %one forth
\newcommand{\prob}[1]{\mathbb{P}(#1)} %Probability of some event
\newcommand{\expe}[1]{\mathbb{E}[#1]} %Expected value for some stochastic variable
\newcommand{\var}[1]{\text{Var}(#1)} %Variance of some stochastic variable
\newcommand{\cov}[1]{\text{Cov}(#1)} %Covariance of some stochastic variables
\newcommand{\corr}[1]{\text{Corr}(#1)} %Correlation of some stochastic variables
\newcommand{\sign}[1]{\text{sign}(#1)} %signum function
\newcommand{\sinc}{\text{sinc}} %sinc function
\newcommand{\diag}{\text{diag}} %diagonal matrix

%----------------------------------------------------------------------------------------------
%Tables
%----------------------------------------------------------------------------------------------

\newcommand{\mc}[2]{\multicolumn{1}{#1}{#2}} %This command is a short for multicolumn
%Usage:
%Instead of \multicolumn{1}{<type>}{<content>}
%write \mc{<type>}{<content>}
%Usecase: When trying to remove vertical lines for only one line in your tabular environment
%multicolumn can come in handy.

%----------------------------------------------------------------------------------------------
% Formating
%----------------------------------------------------------------------------------------------

\newcommand{\sbs}[4]{\begin{minipage}[t!]{#1\linewidth}#3\end{minipage}\begin{minipage}[t!]{#2\linewidth}#4\end{minipage}} %Two minipages dividing the page in two scaled parts
%Usage:
%\sbs{<width of the left, for example 0.45>}{<width on the right, for example 0.45>}
%{
%<content on the left>
%}
%{
%<content on the right>
%}

\newcommand{\sbss}[2]{\sbs{0.45}{0.45}{#1}{#2}} %Two minipages dividing the page in two equal parts
%Usage:
%\sbss
%{
%<content on the left>
%}
%{
%<content on the right>
%}


\newcommand{\sbsss}[3]{\begin{minipage}[t!]{0.42\linewidth}#2\end{minipage}\hspace{#1 ex}\begin{minipage}[t!]{0.42\linewidth}#3\end{minipage}} %Two minipages dividing the page in two equal parts with some user-defined space in [ex] in between
%Usage:
%\sbsss{5}
%{
%<content on the left>
%}
%{
%<content on the right>
%}

%----------------------------------------------------------------------------------------------
%Colors
%----------------------------------------------------------------------------------------------
\definecolor{pantone2128}{RGB}{152,182,228}
\definecolor{pantone2727}{RGB}{48,127,226}
\definecolor{pantone2728}{RGB}{0,71,187}
\definecolor{darkblue}{RGB}{0,35,156}
\definecolor{pantone2757}{RGB}{0,30,96}

%----------------------------------------------------------------------------------------------
%Macros
%----------------------------------------------------------------------------------------------

\newcommand{\statenotation}{
\section{Notation}
\begin{center}
\begin{tabular}{p{0.3\linewidth}p{0.3\linewidth}l}
Operator&Meaning&Example\\
\midrule
\textbf{bold text}&Vector&$\vec{v}$\\
Prepended calligraphic letter&Vector or Matrix represented in the coordinate frame $\mathcal{A}$&$\cvec{A}{v}$\\
Dot above&Time derivative&$\dot{x}$\\
Prime&General derivative&$f'$\\
Double underline&Tensor&$\tens{I}$\\
\midrule
\end{tabular}
\end{center}
}
%Usage:
%Type \statenotation where you'd like to define your notation.

%Checkboxes
\usepackage{pifont}
\newcommand{\cmark}{\ding{51}}%
\newcommand{\xmark}{\ding{55}}%
\newcommand{\done}{\rlap{$\square$}{\raisebox{2pt}{\large\hspace{1pt}\cmark}}%
\hspace{-2.5pt}}
\newcommand{\wontfix}{\rlap{$\square$}{\large\hspace{1pt}\xmark}}
\graphicspath{{./Pictures/}}

\title{Exercise Week 04}
\author{GianAndrea Müller\\ \url{mailto:muellegi@student.ethz}}
\date{\today}

\begin{document}
%Make title page from above supplied information on title, author, etc.
\maketitle

\begin{TFTimeSchedule}
\item 3' Nachbesprechung 
\item 10' Dezimalzahlen im Binärsystem mit Übung
\item 10' Fliesskommasystem $\mathcal{F}^\ast(2,24,-126,127)$
\item 10' Tips zu Fliesskommazahlen
\item 10' Funktionen
\item 15' Pause
\item 10' Funktionsdefinition- und deklaration
\item 10' Pre- and post-conditions
\item 10' Übung zu Funktionen
\end{TFTimeSchedule}

\begin{TFLearningObjectives}
\item Verständnis des Fliesskommasystems
\item Nutzung von Funktionen
\end{TFLearningObjectives}

\begin{frame}{Nachbesprechung}
\begin{itemize}
\item Kommentieren heisst nicht: \textcolor{green}{Dasselbe in grün.}
\item Snippets: Verständnis zeigen!
\end{itemize}
\end{frame}

\note{
\begin{itemize}
\item Ein Kommentar übermittelt Verständnis der Semantik 

und nicht der Syntax. Wieso nicht Was.
\item Man sollte versuchen, den Zweck des Snippets 

zu verstehen. Wichtig für die Prüfung!
\end{itemize}
}

\begin{frame}{Dezimalzahlen im Binärsystem}
\begin{center}
\begin{tabular}{ll|l|l|l|l|l|l|l}
binär:&1&1&1&1&.&1&1&1\\\hline
dezimal:&8&4&2&1&&$\frac{1}{2}$&$\frac{1}{4}$&$\frac{1}{8}$
\end{tabular}
\end{center}
\end{frame}

\begin{TFTwoColumns}[Dezimalzahlen im Binärsystem]
{
\begin{center}
\begin{tabular}{l|l|l}
$x$&$d_i$&$x-d_i$\\\hline
1.934&1&0.934\\
\uncover<2->{9.34&9&0.34\\}
\uncover<3->{3.4&3&0.4\\
4&4&0}
\end{tabular}
\end{center}
}
{
\begin{center}
\uncover<4->{
\begin{tabular}{r|l|l}
$x$&$b_i$&$x-b_i$\\\hline
1.9&1&0.9\\
\uncover<5->{1.8&1&0.8\\}
\uncover<6->{\textbf{1.6}&1&0.6\\}
\uncover<7->{1.2&1&0.2\\}
\uncover<8->{0.4&0&0.8\\}
\uncover<9->{\textbf{1.6}&1&0.6\\}
\uncover<10->{&$\vdots$&}
\end{tabular}}
\end{center}
\uncover<11->{\begin{equation*}
1.1\overline{1100}
\end{equation*}}
}
\end{TFTwoColumns}

\note{
\begin{itemize}
\item Erklärung im Dezimalsystem
\begin{enumerate}
\item Stelle abziehen.
\item Verschiebung mit mal 10.
\end{enumerate}
\item Erklärung im Binärsystem
\begin{enumerate}
\item Stelle abziehen.
\item Verschiebung mit mal 2.
\end{enumerate}
\item Zusatzinfo: Es git also Zahlen, die in bestimmten Zahlensystemen keine \textbf{finite Darstellung} haben!
\end{itemize}
}

\begin{frame}{Exercise 4\_1 $\sim$ 2'}
\begin{block}{Berechne die binäre Darstellung folgender Dezimalzahlen:}
\begin{enumerate}
\item 0.25
\item 11.1
\end{enumerate}
\end{block}
\end{frame}

\begin{TFTwoColumns}[Solution 4\_1]
{
\begin{block}{Lösung 1.}
\begin{tabular}{r|l|l}
$x$&$b_i$&$x-b_i$\\\hline
0.25&0&0.25\\
0.5&0&0.5\\
1&1&0
\end{tabular}
\end{block}
}
{
\begin{block}{Lösung 2.}
\begin{tabular}{r|l|l}
$x$&$b_i$&$x-b_i$\\\hline
0.1&0&0.1\\
0.2&0&0.2\\
0.4&0&0.4\\
0.8&0&0.8\\
1.6&1&0.6\\
1.2&1&0.2\\
&$\vdots$&
\end{tabular}
\end{block}
}
\end{TFTwoColumns}

\begin{frame}{Unser kleines 10bit Fliesskommasystem}
\begin{block}{Beschreibung von Fliesskommasystemen}
$\mathcal{F}(\underbrace{\beta}_\text{Basis $\geq$ 2},\overbrace{p}^\text{Anzahl Stellen$\geq$1},\underbrace{e_{min},e_{max}}_\text{Kleinster und Grösster Exponent})$
\end{block}
\end{frame}

\begin{frame}{Unser kleines 10bit Fliesskommasystem}
\begin{Huge}
\[{\color<2->{red}{0}}{\color<4->{green}{0000}}{\color<3->{blue}{00000}}\]
\end{Huge}
\begin{itemize}
\item<2-> \textcolor{red}{Vorkommastelle}
\item<3-> \textcolor{blue}{Nachkommastellen}
\item<4-> \textcolor{green}{Exponent}
\end{itemize}
\end{frame}

\note{\begin{itemize}
\item Erste Idee: $2.73\cdot10^{12}$
\item Genau ein Stelle, also 1 bit vor dem Komma
\item 5 bits für Nachkommastellen
\item 4 bits für den Exponenten
\item Anwenden des Bezeichnungsschemas
\end{itemize}
}

\begin{frame}{$\mathcal{F}(2,6,0,15)$}
\begin{Huge}
\[{\color{red}{0}}{\color{green}{0000}}{\color{blue}{00000}}\]
\end{Huge}
\begin{itemize}
\item\textcolor{red}{Vorkommastelle}
\item\textcolor{blue}{Nachkommastellen}
\item\textcolor{green}{Exponent}
\end{itemize}
\begin{block}{Beispiele}
$\underbrace{\textcolor{red}{1}.\textcolor{blue}{11111}\cdot 2^{\textcolor{green}{15}}}_\text{Grösste Zahl}\qquad\qquad\underbrace{\textcolor{red}{0}.\textcolor{blue}{00001}\cdot 2^{\textcolor{green}{0}}}_\text{Kleinste Zahl}$
\end{block}
\end{frame}

\note{\begin{itemize}
\item Wir möchten auch negative Exponenten.
\item Interpretation als unsigned int mit Verschiebung ins negative (IEEE 754 Standard), schnellere Rechnungen möglich)
\item 
\end{itemize}}

\begin{frame}{$\mathcal{F}(2,6,-8,7)$}
\begin{Huge}
\[{\color{red}{0}}{\color{green}{0000}}{\color{blue}{00000}}\]
\end{Huge}
\begin{itemize}
\item\textcolor{red}{Vorkommastelle}
\item\textcolor{blue}{Nachkommastellen}
\item\textcolor{green}{Exponent}
\end{itemize}
\begin{block}{Beispiele}
$\underbrace{\textcolor{red}{1}.\textcolor{blue}{11111}\cdot 2^{\textcolor{green}{7}}}_\text{Grösste Zahl}\qquad\qquad\underbrace{\textcolor{red}{0}.\textcolor{blue}{00001}\cdot 2^{\textcolor{green}{-8}}}_\text{Kleinste Zahl}$
\end{block}
\end{frame}

\note{
\begin{itemize}
\item Wir brauchen negative Zahlen!
\item Wir brauchen das erste Bit nicht zwingend wenn wir annehmen, dass es immer 1 ist.
\end{itemize}
}

\begin{frame}{$\mathcal{F}^\ast(2,6,-8,7)$}
\begin{Huge}
\[{\color{red}{0}}{\color{green}{0000}}{\color{blue}{00000}}\]
\end{Huge}
\begin{itemize}
\item\textcolor{red}{Vorzeichenstelle}
\item\textcolor{blue}{Nachkommastellen}
\item\textcolor{green}{Exponent}
\end{itemize}
\begin{block}{Beispiele}

$\underbrace{\textcolor{red}{+}1.\textcolor{blue}{11111}\cdot 2^{\textcolor{green}{7}}}_\text{Grösste Zahl}\qquad\qquad \underbrace{\textcolor{red}{-}1.\textcolor{blue}{11111}\cdot2^{\textcolor{green}{7}}}_\text{Kleinste Zahl}\qquad\qquad\underbrace{\textcolor{red}{+}1.\textcolor{blue}{00000}\cdot2^{\textcolor{green}{-8}}}_\text{Kleinste positive Zahl}$

\end{block}
\end{frame}

\note{\begin{itemize}
\item Wir können 0 nicht mehr darstellen
\item Ein Exponent für spezielle Zahlen: -8 = 0000
\item Und die Nachkommastellen als Codierung für diese Zahlen
\end{itemize}
}
\begin{frame}{$\mathcal{F}^\ast(2,6,-7,7)$}
\begin{Huge}
\[{\color{red}{0}}{\color{green}{0000}}{\color{blue}{00000}}\]
\end{Huge}
\begin{itemize}
\item\textcolor{red}{Vorzeichenstelle}
\item\textcolor{blue}{Nachkommastellen}
\item\textcolor{green}{Exponent}
\end{itemize}
\begin{block}{Beispiele}
\begin{center}
$\underbrace{\textcolor{red}{0}\textcolor{green}{0000}\textcolor{blue}{00000}}_\text{Null}\qquad \underbrace{\textcolor{red}{0}\textcolor{green}{0000}\textcolor{blue}{00001}}_{+\infty}\qquad\underbrace{\textcolor{red}{0}\textcolor{green}{0000}\textcolor{blue}{00010}}_{-\infty}\qquad\underbrace{\textcolor{red}{0}\textcolor{green}{0000}\textcolor{blue}{00011}}_\text{NaN}$
\end{center}
\end{block}
\end{frame}

\note{
\begin{itemize}
\item Wir können 0 nicht mehr darstellen
\item Ein Exponent für spezielle Zahlen: -8 = 0000
\item Und die Nachkommastellen als Codierung für diese Zahlen
\item Jetzt verstehen wir auch 32bit float und 64bit double
\end{itemize}
}

\begin{frame}{Exercise 4\_2 $\sim$ 5'}
\begin{block}{Floating point systems following IEEE 754}
\begin{itemize}
\item float (IEEE 745): $\mathcal{F}^\ast(2,24,-126,127)$
\item double (IEEE 745): $\mathcal{F}^\ast(2,53,-1022,1023)$
\end{itemize}
\begin{enumerate}
\item What is the largest possible normalized single and double precision floating point number?
\item What is the smallest possible normalized single and double precision floatig point number?
\end{enumerate}
\end{block}
\end{frame}

\begin{frame}{Solution 4\_2}
\begin{block}{Floating point systems following IEEE 754}
\begin{enumerate}
\item Smallest normalized number: $2^{e_{min}}$
\item Largest normalized number, float: $+1.11111111111111111111111\cdot2^{127}$
\end{enumerate}
\uncover<2->{\[\left(1-\left(\frac{1}{\beta}\right)^p\right)\beta^{e_{max}+1}\]}
\end{block}
\end{frame}

\begin{frame}{Solution 4\_2}
\begin{block}{$\mathcal{F}^\ast(2,6,-7,7)$}
\begin{enumerate}
\item Grösste positive Zahl: 

\[1.11111\cdot 2^{7}\]
\item Das entspricht:
\[11111100\]
\item Grösste positive Zahl mit 8bit: $2^8-1$
\item Grösste positive Zahl mit 2bit: $2^2-1$
\item Mit variablen:
\begin{align*}(\beta^{e_{max}+1}-1)-(\beta^{e_{max}+1-p}-1) = (\beta^{e_{max}+1}-\beta^{e_{max}+1-p})\\=\left(1-\left(\frac{1}{2}\right)^p\right)\beta^{e_{max}+1}\end{align*}
\end{enumerate}
\end{block}
\end{frame}

\begin{frame}[fragile]{Tipps zu Fliesskommazahlen}
\begin{TFCpp}
//Kein Vergleich gerundeter Zahlen
double  a = 1.1; 
if(100*a == 110) cout<<true<<endl;

//Keine Add. versch. grosser Zahlen
float a = 67108864.0f + 1.0f 
//output: 67108864

//Keine Subtr. aehnlich grosser Zahlen
float x_0 = 0.2; 
//represented as: 0.20000000298
float x_1 = 6*x_0 - 1; //is not 0.2
\end{TFCpp}
\end{frame}

\begin{frame}[fragile]{Funktionsdefinition und -deklaration}
\begin{TFCpp}
void g (...); //declaration of g

void f (...)
{
	g(...);
}

void g (...) // definition of g
{
	f(...);
}
\end{TFCpp}
\end{frame}

\begin{frame}[fragile]{PRE- und POST-Bedingungen}
\begin{TFCpp}
#include <cmath>

int main(){
	// PRE: Value representing angle expressed in radians
	// POST: Cosine of x
	// double cos(double x);
	
	double x = M_PI;
	double result = cos(M_PI);
}
\end{TFCpp}
\end{frame}

\end{document}

